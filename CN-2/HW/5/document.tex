\documentclass{article}
\usepackage{listings}
\usepackage{xcolor}
\usepackage{enumitem}
\usepackage{graphicx}
\usepackage{amssymb}
\usepackage{bytefield}
\usepackage{array}
\usepackage{forest}
\usepackage{float}
\usepackage{fancyhdr} % Custom headers/footers
\usepackage{colortbl}
\usepackage[left=0.6in, right=0.6in, top=1in, bottom=0.9in]{geometry}
\usepackage{indentfirst}
\usepackage{changepage, titlesec}
\usepackage{booktabs}
\usepackage{array}
\usepackage{adjustbox} % for adjustwidth
\usepackage{multicol} % for side-by-side columns
\setlength{\parindent}{1.5em} % Set indentation size (optional)
\usepackage{amsmath} % Required for align environment
\usepackage{xepersian}
\settextfont{Vazirmatn FD}
\setlatintextfont{Noto Serif} 


\pagestyle{fancy}     % Enable fancy headers
\fancyhf{}            % Clear default header/footer
\renewcommand{\headrulewidth}{0pt} % Disable default header line

\fancyhead[L]{\rule{\textwidth}{1pt}} % Manually add one line
\fancyfoot[C]{\thepage} % Page number in the center of the footer

\newcommand{\colorbitbox}[3]{%
	\rlap{\bitbox{#2}{\color{#1}\rule{\width}{\height}}}%
	\bitbox{#2}{#3}}
\definecolor{lightcyan}{rgb}{0.84,1,1}
\definecolor{lightgreen}{rgb}{0.64,1,0.71}
\definecolor{lightred}{rgb}{1,0.7,0.71}

\definecolor{codegreen}{rgb}{0,0.6,0}
\definecolor{codegray}{rgb}{0.5,0.5,0.5}
\definecolor{codepurple}{rgb}{0.58,0,0.82}
\definecolor{backcolour}{rgb}{0.95,0.95,0.92}

\lstdefinestyle{mystyle}{
	backgroundcolor=\color{backcolour},   
	commentstyle=\color{codegreen},
	keywordstyle=\color{magenta},
	numberstyle=\tiny\color{codegray},
	stringstyle=\color{codepurple},
	basicstyle=\ttfamily\footnotesize,
	breakatwhitespace=false,         
	breaklines=true,                 
	captionpos=b,                    
	keepspaces=true,                 
	numbers=left,                    
	numbersep=5pt,                  
	showspaces=false,                
	showstringspaces=false,
	showtabs=false,                  
	tabsize=2
}

\lstset{style=mystyle}

\ExplSyntaxOn
\NewDocumentCommand{\ReverseWords}{m}
{
	\seq_set_split:Nnn \l_tmpa_seq { ~ } { #1 } % Split words by spaces
	\seq_reverse:N \l_tmpa_seq % Reverse the order of words
	\seq_use:Nn \l_tmpa_seq { ~ } % Join words with spaces and output
}
\ExplSyntaxOff

\begin{document}
	\author{ مجتبی ملائی \\ ۴۰۱۳۱۳۸۳ }
	\title{ \huge { تکلیف پنجم}}
	\date{}
	\maketitle
\section{}
\begin{latin}
	\noindent
	$d_{\text{ trans}} = \frac{1500 * 8}{10^9} = 0.012\space \text{ milliseconds}$ \\
	$U = \frac{n*d_{\text{ trans}}}{RTT+d_{\text{ trans}}} = \frac{n*0.012}{30.012} \geq 98\% $ \\
	$n \approx 2451 $
	
\end{latin}
\section{}
از آنجا که ظرفیت لینک بین میزبان A و B برابر با ۱۰۰ مگابیت بر ثانیه است، سرعت ارسال داده از سمت میزبان A از این مقدار بیشتر نمی‌تواند باشد. با این حال، مسئله اصلی در اینجا کنترل جریان TCP است، نه ظرفیت لینک.

میزبان A می‌تواند داده‌ها را با سرعت ۱۰۰ مگابیت بر ثانیه به سوکت TCP خود ارسال کند، ولی میزبان B تنها می‌تواند داده‌ها را با حداکثر سرعت ۵۰ مگابیت بر ثانیه از بافر دریافتی TCP خود بخواند. این بدان معناست که بافر دریافتی میزبان B به مرور پر می‌شود، زیرا داده‌ها سریع‌تر وارد بافر می‌شوند تا از آن خارج شوند. به‌طور دقیق‌تر، بافر با نرخ حدود 50 مگابیت بر ثانیه (50 - 100) پر می‌شود.

زمانی که بافر دریافت پر شود، میزبان B از طریق مکانیزم کنترل جریان در \lr{TCP}، یعنی مقدار صفر شدن فیلد RcvWindow در هدر \lr{TCP}، به میزبان A اطلاع می‌دهد که موقتاً ارسال داده را متوقف کند. در این حالت، میزبان A ارسال را متوقف می‌کند تا زمانی که مجدداً از میزبان B سگمنتی دریافت کند که در آن مقدار RcvWindow بزرگ‌تر از صفر باشد (یعنی فضا در بافر آزاد شده باشد).

در نتیجه، ارسال داده‌ها از میزبان A به صورت توقف و شروع‌های پی‌در‌پی انجام می‌شود و در بلندمدت، سرعت مؤثر انتقال داده‌ها محدود به نرخ خواندن میزبان B از بافر، یعنی ۵۰ مگابیت بر ثانیه خواهد بود.

\section{}	

	دلیل این‌که TCP بلافاصله پس از دریافت نخستین ACK تکراری اقدام به باز‌ارسال نمی‌کند، این است که یک ACK تکراری لزوماً به معنای گم‌شدن یک سگمنت نیست. چنین ACKهایی می‌توانند در شرایط عادی مانند ترتیب خارج شدن بسته‌ها \lr{(packet reordering)} در مسیر، یا ارسال مجدد ACK توسط گیرنده به دلیل دریافت داده‌های بیشتر، ظاهر شوند. اگر TCP به محض دریافت اولین ACK تکراری واکنش نشان دهد، ممکن است منجر به باز‌ارسال‌های غیرضروری شود که خود باعث افزایش ترافیک شبکه و کاهش کارایی کلی می‌گردد.
	
	با منتظر ماندن تا دریافت سه ACK تکراری، احتمال اینکه سگمنتی واقعاً گم شده باشد، به‌طور قابل توجهی افزایش می‌یابد. به‌عبارت دیگر، سه ACK تکراری یک نشانگر قوی از بروز ازدحام یا از دست رفتن بسته است، و نه صرفاً یک اتفاق موقتی در مسیر انتقال.
	
	در نتیجه، این سیاست طراحی باعث می‌شود TCP در برابر خطاهای موقتی و بی‌ضرر مقاوم‌تر باشد و تنها زمانی واکنش نشان دهد که احتمال وجود مشکل واقعی بالا باشد.
	
	
	
	
	
\end{document}