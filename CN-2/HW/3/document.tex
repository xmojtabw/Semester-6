\documentclass{article}
\usepackage{listings}
\usepackage{xcolor}
\usepackage{enumitem}
\usepackage{graphicx}
\usepackage{amssymb}
\usepackage{bytefield}
\usepackage{tikz}
\usetikzlibrary{shapes, positioning}
\usepackage{forest}
\usepackage{fancyhdr} % Custom headers/footers
\usepackage{colortbl}
\usepackage[left=0.6in, right=0.6in, top=1in, bottom=0.9in]{geometry}
\usepackage{indentfirst}
\usepackage{changepage, titlesec}
\usepackage{booktabs}
\usepackage{caption} % Allows using \captionof outside float envir
\usepackage{array}
\usepackage{adjustbox} % for adjustwidth
\usepackage{multicol} % for side-by-side columns
\setlength{\parindent}{1.5em} % Set indentation size (optional)
\usepackage{amsmath} % Required for align environment
\usepackage{xepersian}
\settextfont{Vazirmatn FD}
\setlatintextfont{Noto Serif} 


\pagestyle{fancy}     % Enable fancy headers
\fancyhf{}            % Clear default header/footer
\renewcommand{\headrulewidth}{0pt} % Disable default header line

\fancyhead[L]{\rule{\textwidth}{1pt}} % Manually add one line
\fancyfoot[C]{\thepage} % Page number in the center of the footer

\newcommand{\colorbitbox}[3]{%
	\rlap{\bitbox{#2}{\color{#1}\rule{\width}{\height}}}%
	\bitbox{#2}{#3}}
\definecolor{lightcyan}{rgb}{0.84,1,1}
\definecolor{lightgreen}{rgb}{0.64,1,0.71}
\definecolor{lightred}{rgb}{1,0.7,0.71}

\definecolor{codegreen}{rgb}{0,0.6,0}
\definecolor{codegray}{rgb}{0.5,0.5,0.5}
\definecolor{codepurple}{rgb}{0.58,0,0.82}
\definecolor{backcolour}{rgb}{0.95,0.95,0.92}

\lstdefinestyle{mystyle}{
	backgroundcolor=\color{backcolour},   
	commentstyle=\color{codegreen},
	keywordstyle=\color{magenta},
	numberstyle=\tiny\color{codegray},
	stringstyle=\color{codepurple},
	basicstyle=\ttfamily\footnotesize,
	breakatwhitespace=false,         
	breaklines=true,                 
	captionpos=b,                    
	keepspaces=true,                 
	numbers=left,                    
	numbersep=5pt,                  
	showspaces=false,                
	showstringspaces=false,
	showtabs=false,                  
	tabsize=2
}

\lstset{style=mystyle}

\ExplSyntaxOn
\NewDocumentCommand{\ReverseWords}{m}
{
	\seq_set_split:Nnn \l_tmpa_seq { ~ } { #1 } % Split words by spaces
	\seq_reverse:N \l_tmpa_seq % Reverse the order of words
	\seq_use:Nn \l_tmpa_seq { ~ } % Join words with spaces and output
}
\ExplSyntaxOff

\begin{document}
	\title{\huge تکلیف سوم}
	\author{مجتبی ملایی \\ ۴۰۱۳۱۳۸۳}
	\maketitle
	
	
	
	\section{P1}
	\begin{enumerate}
		\item  ‎  ‎ 
		
		\begin{tabular}[h!]{|c|c|}
			\hline
			    Interface & Destination \\
			\hline
			 \lr{3} & \lr{H3}  \\
			\hline
		\end{tabular}
		\item
		 خیر نمیتوان، زیرا عملیات forwarding فقط بر اساس \textbf{مقصد} انجام میشود. 

	\end{enumerate}
	
	\section{P2}
	\begin{enumerate}
		\item 
		خیر ممکن نیست. زیرا در روش گذرگاه(bus) در هر لحظه \textbf{فقط یک بسته} میتواند از bus عبور کند.
		\item 
		خیر زیرا حافظه سوئیچ امکان خواندن و نوشتن \textbf{همزمان} را ندارد.     فقط \textbf{یک عملیات} در یک زمان میتواند انجام شود.
		\item 
خیر ممکن نیست. چون آن گذرگاه خروجی در یک لحظه فقط میتواند به \textbf{یکی} از گذرگاه های ورودی وصل یاشد.
 
	\end{enumerate}
	
	
	\section{P3}
	اگر روتر N پورت داشته باشد، و ارسال هر بسته از پورت ورودی به پورت خروجی توسط روتر به اندازه D طول بکشد، حداکثر زمانی که بسته باید منتظر بماند:
	\begin{itemize}
		
		\item [\textbf{(a}] 
		\textbf{با استفاده از روش :memory} چون در روش حافظه، در هر زمان فقط یک بسته میتواند پردازش شود، در بدترین حالت بسته باید منتظر بماند تا بسته هایی که از N-1 پورت دیگر آمده اند پردازش شوند. بنابراین به اندازه \( D.(N-1) \) باید منتظر بماند.
	
		\item [\textbf{(b}] 
		\textbf{با استفاده از روش :bus} در این روش هم فقط یک بسته همزمان میتواند از bus عبور کند. پس در بدترین حالت باید به اندازه \( D.(N-1) \) منتظر بماند.
		
		\item [\textbf{(c}]
		\textbf{با استفاده از روش :cross-bar} چون خروجی هر بسته متفاوت است، حتی در بدترین حالت هر پورت خروجی فقط به یکی از پورت های ورودی وصل می‌شود. بنابراین همه بسته ها همزمان به پورت های خروجی خود میروند. در نتیجه بسته دچار تاخیر نمی‌شود.
	\end{itemize}
	
	\section{P4}
در شکل زیر دو حالت برای انتقال بسته ها وچود دارد.
\begin{enumerate}
	\item حالت اول:
	\begin{enumerate}
		\item [\lr{:slot 1}] 
		بسته X از پورت اول و بسته Y از پورت دوم.
		\item [\lr{:slot 2}]  
		بسته X از پورت دوم و بسته Y از پورت سوم.
		\item [\lr{:slot 3}]  
		بسته Z از پورت سوم.
	\end{enumerate}
	\item حالت دوم:
	\begin{enumerate}
		\item [\lr{:slot 1}]  
		بسته X از پورت اول و بسته Y از پورت سوم.
		\item [\lr{:slot 2}]  
		بسته Y از پورت دوم و بسته Z از پورت سوم.
		\item [\lr{:slot 3}] 
		بسته X از پورت دوم.
	\end{enumerate}
	بنابراین در بهترین و در بدترین حالت به \textbf{۳ اسلات زمانی} نیاز داریم. 
\end{enumerate}
	
	
  	
\end{document}







