\documentclass{article}
\usepackage{listings}
\usepackage{xcolor}
\usepackage{enumitem}
\usepackage{graphicx}
\usepackage{amssymb}
\usepackage{bytefield}
\usepackage{array}
\usepackage{forest}
\usepackage{float}
\usepackage{fancyhdr} % Custom headers/footers
\usepackage{colortbl}
\usepackage[left=0.6in, right=0.6in, top=1in, bottom=0.9in]{geometry}
\usepackage{indentfirst}
\usepackage{changepage, titlesec}
\usepackage{booktabs}
\usepackage{array}
\usepackage{adjustbox} % for adjustwidth
\usepackage{multicol} % for side-by-side columns
\setlength{\parindent}{1.5em} % Set indentation size (optional)
\usepackage{amsmath} % Required for align environment
\usepackage{xepersian}
\settextfont{Vazirmatn FD}
\setlatintextfont{Noto Serif} 


\pagestyle{fancy}     % Enable fancy headers
\fancyhf{}            % Clear default header/footer
\renewcommand{\headrulewidth}{0pt} % Disable default header line

\fancyhead[L]{\rule{\textwidth}{1pt}} % Manually add one line
\fancyfoot[C]{\thepage} % Page number in the center of the footer

\newcommand{\colorbitbox}[3]{%
	\rlap{\bitbox{#2}{\color{#1}\rule{\width}{\height}}}%
	\bitbox{#2}{#3}}
\definecolor{lightcyan}{rgb}{0.84,1,1}
\definecolor{lightgreen}{rgb}{0.64,1,0.71}
\definecolor{lightred}{rgb}{1,0.7,0.71}

\definecolor{codegreen}{rgb}{0,0.6,0}
\definecolor{codegray}{rgb}{0.5,0.5,0.5}
\definecolor{codepurple}{rgb}{0.58,0,0.82}
\definecolor{backcolour}{rgb}{0.95,0.95,0.92}

\lstdefinestyle{mystyle}{
	backgroundcolor=\color{backcolour},   
	commentstyle=\color{codegreen},
	keywordstyle=\color{magenta},
	numberstyle=\tiny\color{codegray},
	stringstyle=\color{codepurple},
	basicstyle=\ttfamily\footnotesize,
	breakatwhitespace=false,         
	breaklines=true,                 
	captionpos=b,                    
	keepspaces=true,                 
	numbers=left,                    
	numbersep=5pt,                  
	showspaces=false,                
	showstringspaces=false,
	showtabs=false,                  
	tabsize=2
}

\lstset{style=mystyle}

\ExplSyntaxOn
\NewDocumentCommand{\ReverseWords}{m}
{
	\seq_set_split:Nnn \l_tmpa_seq { ~ } { #1 } % Split words by spaces
	\seq_reverse:N \l_tmpa_seq % Reverse the order of words
	\seq_use:Nn \l_tmpa_seq { ~ } % Join words with spaces and output
}
\ExplSyntaxOff

\begin{document}
	\author{ مجتبی ملائی \\ ۴۰۱۳۱۳۸۳ }
	\title{ \huge { تکلیف هشتم}}
	\date{}
	\maketitle
	\section{}
	
	\begin{enumerate}
		\item 
		چون ۵۰٪ پهنای باند یک و ۲۵٪ دو و ۲۵٪ دیگر ۴ است پس می‌تواند به این شکل باشد:
		\begin{latin}
					1213 1213 ...
		\end{latin}

		\item 
		در این حالت پهنای باند اضافی بین بقیه به اشتراک گذاشته می‌شود. 
		پهنای باند اولی می‌شود:
		$\frac{0.5}{0.5+0.25}=\frac{2}{3}$
		
		پهنای باند دومی می‌شود:
		$\frac{0.25}{0.5+0.25}=\frac{1}{3}$
		
		پس می‌تواند به این شکل باشد:
		\begin{latin}
			121 121 121 ...
		\end{latin}
	\end{enumerate}
	\footnotesize
	 \textit{ فاصله‌ها فقط برای خوانایی بهتر هستند.}
	 
\section{}
\begin{enumerate}
	\item 
	\begin{latin}
		
		\[
		\begin{array}{|c|c|c|c|}
			\hline
			\textbf{Packet} & \textbf{Arrival Time} & \textbf{Departure Time} & \textbf{Delay (Departure - Arrival)} \\
			\hline
			2 & 0 & 1 & 1 \\
			3 & 1 & 2 & 1 \\
			4 & 1 & 3 & 2 \\
			6 & 2 & 4 & 2 \\
			5 & 3 & 5 & 2 \\
			7 & 3 & 6 & 3 \\
			8 & 5 & 7 & 2 \\
			9 & 5 & 8 & 3 \\
			10 & 7 & 9 & 2 \\
			11 & 8 & 10 & 2 \\
			12 & 8 & 11 & 3 \\
			\hline
		\end{array}
		\]
		
	\end{latin}
	مجموع تاخیر برابر است با:
	\[
	1 + 1 + 2 + 2 + 2 + 3 + 2 + 3 + 2 + 2 + 3 = 23
	\]
	میانگین تاخیر:
	\begin{latin}
	\[
	\frac{23}{11} \approx 2.09 \space \text{    time units}
	\]
		\end{latin}
		
		\item 
		‎ ‎  ‎
		\begin{latin}
\begin{center}
	\begin{tabular}{|c|c|c|c|}
		\hline
		\textbf{Packet} & \textbf{Arrival Time} & \textbf{Departure Time} & \textbf{Delay} \\
		\hline
		2 & 0 & 2 & 2 \\
		3 & 1 & 1 & 0 \\
		4 & 1 & 6 & 5 \\
		5 & 3 & 3 & 0 \\
		6 & 2 & 7 & 5 \\
		7 & 3 & 4 & 1 \\
		8 & 5 & 9 & 4 \\
		9 & 5 & 5 & 0 \\
		10 & 7 & 10 & 3 \\
		11 & 8 & 8 & 0 \\
		12 & 8 & 11 & 3 \\
		\hline
	\end{tabular}
\end{center}
		\end{latin}
		میانگین برابر است با:
	\[
		\text{Average delay} = \frac{2 + 0 + 5 + 0 + 5 + 1 + 4 + 0 + 3 + 0 + 3}{11} = \frac{23}{11} \approx 2.09
	\]
		
		
\item 
 ‎  ‎
\begin{latin}
	\begin{center}
		\begin{tabular}{|c|c|c|c|}
			\hline
			\textbf{Packet} & \textbf{Arrival Time} & \textbf{Departure Time} & \textbf{Delay} \\
			\hline
			2 & 0 & 1 & 1 \\
			3 & 1 & 3 & 2 \\
			4 & 1 & 2 & 1 \\
			5 & 3 & 4 & 1 \\
			6 & 2 & 5 & 3 \\
			7 & 3 & 6 & 3 \\
			8 & 5 & 7 & 2 \\
			9 & 5 & 9 & 4 \\
			10 & 7 & 11 & 4 \\
			11 & 8 & 8 & 0 \\
			12 & 8 & 10 & 2 \\
			\hline
		\end{tabular}
	\end{center}
	\[
	\text{Average delay} = \frac{1 + 2 + 1 + 1 + 3 + 3 + 2 + 4 + 4 + 0 + 2}{11} = \frac{23}{11} \approx 2.09
	\]
\end{latin}		

\item 		
		\begin{latin}
			\begin{center}
				\begin{tabular}{|c|c|c|c|}
					\hline
					\textbf{Packet} & \textbf{Arrival Time} & \textbf{Departure Time} & \textbf{Delay} \\
					\hline
					2  & 0 & 1  & 1 \\
					3  & 1 & 2  & 1 \\
					4  & 1 & 4  & 3 \\
					5  & 3 & 3  & 0 \\
					6  & 2 & 7  & 5 \\
					7  & 3 & 5  & 2 \\
					8  & 5 & 9  & 4 \\
					9  & 5 & 6  & 1 \\
					10 & 7 & 10 & 3 \\
					11 & 8 & 8  & 0 \\
					12 & 8 & 11 & 3 \\
					\hline
				\end{tabular}
			\end{center}
			
			\[
			\text{Average delay} = \frac{1 + 1 + 3 + 0 + 5 + 2 + 4 + 1 + 3 + 0 + 3}{11} = \frac{23}{11} \approx 2.09
			\]
		\end{latin}
		\item 
		بدون توجه به نوع الگوریتم زمانبندی، میانگین تاخیر هواره ثابت است. هرجور زمانبندی کنیم، نهایتا تاخیر یک بسته را زیاد کرده و تاخیر دیگری را کم می‌کنیم چون همواره فقط یک بسته درحال پردازش است. 
\end{enumerate}

	 
\end{document}
