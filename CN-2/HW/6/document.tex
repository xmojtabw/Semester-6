\documentclass{article}
\usepackage{listings}
\usepackage{xcolor}
\usepackage{enumitem}
\usepackage{graphicx}
\usepackage{amssymb}
\usepackage{bytefield}
\usepackage{array}
\usepackage{forest}
\usepackage{float}
\usepackage{fancyhdr} % Custom headers/footers
\usepackage{colortbl}
\usepackage[left=0.6in, right=0.6in, top=1in, bottom=0.9in]{geometry}
\usepackage{indentfirst}
\usepackage{changepage, titlesec}
\usepackage{booktabs}
\usepackage{array}
\usepackage{adjustbox} % for adjustwidth
\usepackage{multicol} % for side-by-side columns
\setlength{\parindent}{1.5em} % Set indentation size (optional)
\usepackage{amsmath} % Required for align environment
\usepackage{xepersian}
\settextfont{Vazirmatn FD}
\setlatintextfont{Noto Serif} 


\pagestyle{fancy}     % Enable fancy headers
\fancyhf{}            % Clear default header/footer
\renewcommand{\headrulewidth}{0pt} % Disable default header line

\fancyhead[L]{\rule{\textwidth}{1pt}} % Manually add one line
\fancyfoot[C]{\thepage} % Page number in the center of the footer

\newcommand{\colorbitbox}[3]{%
	\rlap{\bitbox{#2}{\color{#1}\rule{\width}{\height}}}%
	\bitbox{#2}{#3}}
\definecolor{lightcyan}{rgb}{0.84,1,1}
\definecolor{lightgreen}{rgb}{0.64,1,0.71}
\definecolor{lightred}{rgb}{1,0.7,0.71}

\definecolor{codegreen}{rgb}{0,0.6,0}
\definecolor{codegray}{rgb}{0.5,0.5,0.5}
\definecolor{codepurple}{rgb}{0.58,0,0.82}
\definecolor{backcolour}{rgb}{0.95,0.95,0.92}

\lstdefinestyle{mystyle}{
	backgroundcolor=\color{backcolour},   
	commentstyle=\color{codegreen},
	keywordstyle=\color{magenta},
	numberstyle=\tiny\color{codegray},
	stringstyle=\color{codepurple},
	basicstyle=\ttfamily\footnotesize,
	breakatwhitespace=false,         
	breaklines=true,                 
	captionpos=b,                    
	keepspaces=true,                 
	numbers=left,                    
	numbersep=5pt,                  
	showspaces=false,                
	showstringspaces=false,
	showtabs=false,                  
	tabsize=2
}

\lstset{style=mystyle}

\ExplSyntaxOn
\NewDocumentCommand{\ReverseWords}{m}
{
	\seq_set_split:Nnn \l_tmpa_seq { ~ } { #1 } % Split words by spaces
	\seq_reverse:N \l_tmpa_seq % Reverse the order of words
	\seq_use:Nn \l_tmpa_seq { ~ } % Join words with spaces and output
}
\ExplSyntaxOff

\begin{document}
	\author{ مجتبی ملائی \\ ۴۰۱۳۱۳۸۳ }
	\title{ \huge { تکلیف ششم}}
	\date{}
	\maketitle
	
	\section{}
	
	\begin{enumerate}
		\item 
		در ۱ تا ۶ و ۲۳ تا ۲۶
		\item 
		در ۶ تا ۱۶ و ۱۷ تا ۲۲
		\item 
		\ReverseWords{triple duplicate ack} رخ داده است زیرا cwnd یک نشده است.
\item 
		timoeout رخ داده است زیرا cwnd یک شده است.
		
\item 
۳۲ است. زیرا در آنجا \ReverseWords{congestion avoidance} شروع شده است.
\item 
وقتی در راند ۱۶ \ReverseWords{triple duplicate ack} رخ می‌دهد sstresh برابر cwnd/2  می‌شود. یعنی بشود 42/2 که برابر است با 21. 
\item 
وقتی در راند دوم timeout رخ میدهد، sstresh برابر با cwnd/2 میشود که برابر است با 29/2 که می‌شود ۱۴ (حد پایین باید حساب کنیم).
\item 
\begin{tabular}{|c|c|c|c|c|c|c|c|}
	\hline
راند 	& ۱ &۲  & ۳ & ۴ &۵  & ۶ & ۷ \\
	\hline
آخرین پکت ارسالی	& ۱ & ۳ & ۷ & ۱۵ & ۳۱ & ۶۳ & ۹۶ \\
	\hline
اندازه پنجره	& ۱ & ۲ & ۴ & ۸ & ۱۶ & ۳۲ & ۳۳ \\
	\hline 
\end{tabular}
	همانطور که دیده میشود، در راند هفتم پکت ۷۰ ارسال میشود.
	\item 
	میدانیم اگر چنین اتفاقی رخ دهد، \lr{sstresh = cwnd/2 } و \lr{cwnd = sstresh + 3 * MSS} است. و چون cwnd قبلی برابر است با ۸ پس sstresh میشود ۴ و cwnd میشود ۷. 
	\item 
	اگر Tahoe استفاده شود sstresh برابر با 21 میشود (همانند حالت قبل) اما این بار وارد   \ReverseWords{slow start} می شویم. در واقع در راند ۱۷ سایز پنجره برابر با ۱ و در راند  ۱۹ برابر با ۴ خواهد بود. 
	
	\item 
	\begin{tabular}{|c|c|c|c|c|c|c|}
		\hline
		راند 	& ۱۷ &  ۱۸ & ۱۹ & ۲۰ &۲۱  & ۲۲  \\
		\hline
		آخرین پکت ارسالی	& ۱ & ۳ & ۷ & ۱۵ & ۳۱ & ۵۲  \\
		\hline
		اندازه پنجره	& ۱ & ۲ & ۴ & ۸ & ۱۶ & ۲۱  \\
		\hline 
	\end{tabular}
	بنابرانی ۵۲ پکت ارسال خواهند شد.
	\end{enumerate}
	
	\section{}
	\begin{enumerate}
		\item 
		چون به ازای هر دسته از ACK ها فقط یک MSS زیاد میشود، بنابراین بعد از ارسال کامل هر پنجره و دریافت ACK آنها (یک RTT) سایز پنجره یک MSS زیاد میشود. پس بعد از گذشت ۶ RTT اندازه پنجره از 6 به ۱۲ می‌رسد.
		\item 
		 	\begin{tabular}{|c|c|c|c|c|c|c|c|}
		 	\hline
		 	راند 	& ۶ &  ۷ & ۸ & ۹ & ۱۰  & ۱۱ & ۱۲  \\
		 	\hline
		 	اندازه پنجره	& ۶ &  ۷ & ۸ & ۹ & ۱۰  & ۱۱ & ۱۲  \\
		 	\hline 
		 \end{tabular}
		 
		 چون در هر راند کل پنجره ارسال شده است پس داریم:
		 \(
		 thr = \frac{6+7+8+9+10+11+12}{6} = 8.5 \space \space MSS/RTT
		 \)
	\end{enumerate}
	\section{}
\begin{enumerate}
	\item 
	اگر W حداکثر اندازه پنجره باشد، می‌دانیم که 
	\(
	\frac{W*MSS}{RTT} = R
	\)
	بنابراین داریم:
	\(
	\frac{W*1500*8}{150*10^{-3}} = 10*10^6 
	\)
	درنتیجه W برابراست با 125000.
	\item 
	چون در Reno اندازه پنجره هر بار پس از congestion نصف میشود و دوباره به حداکثر خود  می‌رسد پس بین W/2 و W قرار دارد و به دلیل \ReverseWords{congestion avoidance} به صورت خطی  در این بازه حرکت می‌کند. پس میانگین اندازه پنجره برابر است با \lr{0.75W} یعنی ۹۳۷۵۰. حالا مدانیم که throughput برابر است با  
	\(
	\frac{W*MSS}{RTT} 
	\)
	در نتیجه برابر می‌شود با \lr{7.5Gbps}
	
	\item 
	چون در هر RTT یک سگمنت به اندازه پنجره اضافه میشود پس نیاز داریم W/2 * RTT بگذرد تا به حداکثر برسد. درنتیجه داریم 
	\(
	62500×0.15=9375 seconds
	\)
	که به طور تقریبی برابر است با ۱۵۶ دقیقه.
\end{enumerate}
\end{document}