\documentclass{article}
\usepackage{listings}
\usepackage{xcolor}
\usepackage{enumitem}
\usepackage{graphicx}
\usepackage{amssymb}
\usepackage{bytefield}
\usepackage{forest}
\usepackage{float}
\usepackage{fancyhdr} % Custom headers/footers
\usepackage{colortbl}
\usepackage[left=0.6in, right=0.6in, top=1in, bottom=0.9in]{geometry}
\usepackage{indentfirst}
\usepackage{changepage, titlesec}
\usepackage{booktabs}
\usepackage{array}
\usepackage{listings}
\usepackage{minted}
\usepackage{xcolor}

\usepackage{hyperref}
\usepackage{adjustbox} % for adjustwidth
\usepackage{multicol} % for side-by-side columns
\setlength{\parindent}{1.5em} % Set indentation size (optional)
\usepackage{amsmath} % Required for align environment
\usepackage{xepersian}
\settextfont{Vazirmatn FD}
\setlatintextfont{Noto Serif} 



\lstset{backgroundcolor=\color{gray!20}, basicstyle=\ttfamily}
\newcommand{\code}[1]{\colorbox{gray!10}{\texttt{#1}}}

\pagestyle{fancy}     % Enable fancy headers
\fancyhf{}            % Clear default header/footer
\renewcommand{\headrulewidth}{0pt} % Disable default header line

\fancyhead[L]{\rule{\textwidth}{1pt}} % Manually add one line
\fancyfoot[C]{\thepage} % Page number in the center of the footer

\newcommand{\colorbitbox}[3]{%
	\rlap{\bitbox{#2}{\color{#1}\rule{\width}{\height}}}%
	\bitbox{#2}{#3}}
\definecolor{lightcyan}{rgb}{0.84,1,1}
\definecolor{lightgreen}{rgb}{0.64,1,0.71}
\definecolor{lightred}{rgb}{1,0.7,0.71}

\definecolor{codegreen}{rgb}{0,0.6,0}
\definecolor{codegray}{rgb}{0.5,0.5,0.5}
\definecolor{codepurple}{rgb}{0.58,0,0.82}
\definecolor{backcolour}{rgb}{0.95,0.95,0.92}

\lstdefinestyle{mystyle}{
	backgroundcolor=\color{backcolour},   
	commentstyle=\color{codegreen},
	keywordstyle=\color{magenta},
	numberstyle=\tiny\color{codegray},
	stringstyle=\color{codepurple},
	basicstyle=\ttfamily\footnotesize,
	breakatwhitespace=false,         
	breaklines=true,                 
	captionpos=b,                    
	keepspaces=true,                 
	numbers=left,                    
	numbersep=5pt,                  
	showspaces=false,                
	showstringspaces=false,
	showtabs=false,                  
	tabsize=2
}

\lstset{style=mystyle}

\ExplSyntaxOn
\NewDocumentCommand{\ReverseWords}{m}
{
	\seq_set_split:Nnn \l_tmpa_seq { ~ } { #1 } % Split words by spaces
	\seq_reverse:N \l_tmpa_seq % Reverse the order of words
	\seq_use:Nn \l_tmpa_seq { ~ } % Join words with spaces and output
}
\ExplSyntaxOff


\begin{document}
	\author{ مجتبی ملائی \\ ۴۰۱۳۱۳۸۳ 		}
	\title{ \huge { پروژه دوم}}
	\date{}
	\maketitle
\section{تحلیل ترافیک شبکه}

در این سوال از توپولوژی signel با ۳ میزبان استفاده شد و سویچ به یک کنترل خارجی با پروتکل OpenFlow1.3 متصل گشت. بسته های میزبان اول و همچنین OpenFlow ذخیره شده اند. پس از اجرا، میزبان \lr{h1} از میزبان \lr{h2} پینگ می‌گیرد.
 
\noindent
در میزبان اول بسته های زیر دیده می‌شوند:
\begin{itemize}
	\item \lr{ICMPv6 Router Solicitation}: بسته هایی هستند که میزبان ها برای پیدا کردن روتر در پروتکل \lr{IPv6} ارسال می‌کنند. (در این سوال کابربردی ندارند.) 
	\item :ARP برای پیدا کردن mac-address متناظر با یک IP در یک شبکه محلی
	\item :ICMP برای پینگ گرفتن
\end{itemize}
\noindent
در کنترل خارجی بسته های زیر دیده می‌شود:
\begin{itemize}
	\item \lr{TCP}: ارتباط قابل اطمینان بین کنترلر و سوئیچ
	\item \lr{OpenFlow}: ارتباط و انتقال دستورات و اطلاعات میان کنترلر و سوئیچ
\end{itemize}

\begin{enumerate}[label=\textbf{\Alph*)}]
	\item 
	در پروتکل OpenFlow، هدف از تبادل پیام‌های \ReverseWords{Feature Request} و \ReverseWords{Feature Reply} بین سوئیچ و کنترلر، شناسایی قابلیت‌ها و ویژگی‌های سوئیچ توسط کنترلر است. این تبادل به کنترلر اجازه می‌دهد تا از امکانات سخت‌افزاری و نرم‌افزاری سوئیچ آگاه شود و بر اساس آن، تصمیم‌گیری‌های مناسب انجام دهد.برخی از این اطلاعات شامل: 
	\begin{itemize}
		\item شناسه سوئیچ \ReverseWords{(Datapath ID)}
		\item 
تعداد جدول‌ها \ReverseWords{:(nـtables)} تعداد جدول‌هایی که سوئیچ پشتیبانی می‌کند
		\item 
		قابلیت‌ها 
	\end{itemize}
	\begin{figure}[H]
		\centering
		\includegraphics[width=0.7\linewidth]{screenshot001}
		%\caption{بسته های بین }
		\label{fig:screenshot001}
	\end{figure}
	\begin{figure}[H]
		\centering
		\includegraphics[width=0.5\linewidth]{screenshot002}
		\label{fig:screenshot002}
	\end{figure}
	
	
	\item در این دو سناریو بسته به کنترلر ارسال می‌شود.
	\begin{itemize}
		\item \lr{:\texttt{OFPR\_NO\_MATCH(0)}} وقتی هیچ Flow ای برای بسته در جدول وجود نداشته باشد و اصطلاحا miss رخ دهد.
		\item  \lr{:\texttt{OFPR\_ACTION(1)}} وقتی در Flow متناظر با آن بسته action آن ارسال به کنترلر باشد.
	\end{itemize}
	\item در ارسال بسته های ICMP از پروتکل IP استفاده می‌شود. ابتدا با استفاده از ARP به mac-address مورد نظر دست پیدا می‌کند. سپس بسته های ICMP در بسته های IP  و سپس در فریم Ethernet قرار می‌گیرند. 
	در این پروتکل مقدار type در بسته ها ICMP برابر با 8 قرار می‌گیرد که به معنی \ReverseWords{Echo Request} است. و در پاسخ این فیلد مقدار \lr{0} می‌گیرد که به معنای \ReverseWords{Echo Reply} است. از فیلد ident برای متمایز کردن پینگ بین پروسس های متفاوت استفاده می‌شود و از seq برای ترتیب و متمایز کردن بسته ها استفاده می‌شود. همچنین هر بسته دارای timestamp می‌باشد.
	\begin{figure}[H]
		\centering
		\includegraphics[width=0.7\linewidth]{screenshot003}
		\label{fig:screenshot003}
	\end{figure}
	\begin{figure}[H]
		\centering
		\includegraphics[width=0.4\linewidth,height=0.2\textheight]{screenshot004}
		\label{fig:screenshot004}
	\end{figure}
	
\end{enumerate}

\section{اجرای توپولوژی‌های مختلف در Mininet}
\begin{enumerate}
	\item \code{\ReverseWords{sudo mn --topo single,5 --mac --switch ovsk}}:
	این دستور از توپولوژی سینگل (یک سوئیچ مرکزی) و ۵ میزبان که به آن متصل هستند استفاده می‌کند. 
	mac-address ها به صورت اتوماتیک تنظیم می‌شوند و از سويچ مجازی در کرنل برای سویچ استفاده می‌شود. 
	\item \code{\ReverseWords{sudo mn --topo single,5 --controller remote -x}}:
	این دستور از توپولوژی سینگل (یک سوئیچ مرکزی) و ۵ میزبان که به آن متصل هستند استفاده می‌کند. آپشن کنترلر باعث می‌شود تا سویچ مرکزی به یک کنترلر خارجی SDN بر روی پورت ۶۶۳۳ یا ۶۶۵۳ طبق پروتکل OpenFlow متصل شود. اپشن x نیز به ازای هر دستگاه (میزبان، سوئیچ، کنترلر) یک shell ایجاد می‌کند. 
	
	\item \code{\ReverseWords{sudo mn --topo tree,5 --mac --arp}}:
		این دستور از توپولوژی درخت به شکل باینری و عمق ۵ (فقط سوئیچ) و میزبان ها که به برگ ها وصل شده اند استفاده می‌کند. ۳۲ میزبان و ۳۱ سوئیچ.  
		mac-address ها به صورت اتوماتیک تنظیم می‌شوند. همچنین مقدار های arp-entry کل شبکه برای تمام میزبان ها از قبل تنظیم می‌شود.  
	\item \code{\ReverseWords{sudo mn --topo linear --controller=remote,ip=127.0.0.1,port=6633}}:
	این دستور از توپولوژی خطی (دو میزبان و دو سوئیچ) استفاده می‌کند. 
	آپشن کنترلر باعث می‌شود تا سوئیچ ها به یک کنترلر خارجی SDN بر روی آدرس IP و پورت مشخص شده متصل شوند. 
\end{enumerate}
	\begin{enumerate}
		\item 
		به ترتیب:‌ سینگل‌، سینگل، درختی و خطی
		\item 
		 \begin{itemize}
		 	\item \textbf{\code{\ReverseWords{--mac}}}:
		 	این دستور mac-address ها را برای میزبان ها به صورت مشخص از \code{\lr{00:00:00:00:00:01}} شروع به تخصیص می‌کند. مثلا mac-address میزبان \lr{h10} برابر خواهد بود با  \code{\lr{00:00:00:00:00:10}} این کار دیباگ کردن شبکه را آسانتر می‌کند.
		 	\item \textbf{\code{\ReverseWords{--arp}}}:
		 	این آپشن arp-entity ها را به صورت خودکار برای همه میزبان ها پر می‌کند. بنابراین نیازی به اجرای پروتکل arp نیست.  
		 	
		 \end{itemize}
	\end{enumerate} 
	
\section{}
\begin{enumerate}
	\item 
از همه میزبان ها به یکدیگر پینگ گرفته شد و نشان میدهد شبکه به درستی کار می‌کند. 
\begin{latin}
\begin{lstlisting}
sudo python tree_topo.py 2   
*** Creating network
*** Adding controller
*** Adding hosts:
h1 h2 h3 h4 
*** Adding switches:
s1 s2 s3 
*** Adding links:
(s1, s2) (s1, s3) (s2, h1) (s2, h2) (s3, h3) (s3, h4) 
*** Configuring hosts
h1 h2 h3 h4 
*** Starting controller
c0 
*** Starting 3 switches
s1 s2 s3 ...
*** Starting CLI:
mininet> h1 ping h2
PING 10.0.0.2 (10.0.0.2) 56(84) bytes of data.
64 bytes from 10.0.0.2: icmp_seq=1 ttl=64 time=3.58 ms
64 bytes from 10.0.0.2: icmp_seq=2 ttl=64 time=0.997 ms
^C
--- 10.0.0.2 ping statistics ---
2 packets transmitted, 2 received, 0% packet loss, time 1002ms
rtt min/avg/max/mdev = 0.997/2.288/3.580/1.291 ms
mininet> h1 ping h3
PING 10.0.0.3 (10.0.0.3) 56(84) bytes of data.
64 bytes from 10.0.0.3: icmp_seq=1 ttl=64 time=5.76 ms
64 bytes from 10.0.0.3: icmp_seq=2 ttl=64 time=2.11 ms
^C
--- 10.0.0.3 ping statistics ---
2 packets transmitted, 2 received, 0% packet loss, time 1002ms
rtt min/avg/max/mdev = 2.105/3.931/5.758/1.826 ms
mininet> h1 ping h4
PING 10.0.0.4 (10.0.0.4) 56(84) bytes of data.
64 bytes from 10.0.0.4: icmp_seq=1 ttl=64 time=5.76 ms
64 bytes from 10.0.0.4: icmp_seq=2 ttl=64 time=2.06 ms
64 bytes from 10.0.0.4: icmp_seq=3 ttl=64 time=0.171 ms
^C
--- 10.0.0.4 ping statistics ---
3 packets transmitted, 3 received, 0% packet loss, time 2004ms
rtt min/avg/max/mdev = 0.171/2.666/5.764/2.322 ms
mininet> h2 ping h3
PING 10.0.0.3 (10.0.0.3) 56(84) bytes of data.
64 bytes from 10.0.0.3: icmp_seq=1 ttl=64 time=5.84 ms
64 bytes from 10.0.0.3: icmp_seq=2 ttl=64 time=2.18 ms
64 bytes from 10.0.0.3: icmp_seq=3 ttl=64 time=0.094 ms
^C
--- 10.0.0.3 ping statistics ---
3 packets transmitted, 3 received, 0% packet loss, time 2003ms
rtt min/avg/max/mdev = 0.094/2.705/5.837/2.373 ms
mininet> h2 ping h4
PING 10.0.0.4 (10.0.0.4) 56(84) bytes of data.
64 bytes from 10.0.0.4: icmp_seq=1 ttl=64 time=5.10 ms
64 bytes from 10.0.0.4: icmp_seq=2 ttl=64 time=1.66 ms
^C
--- 10.0.0.4 ping statistics ---
2 packets transmitted, 2 received, 0% packet loss, time 1001ms
rtt min/avg/max/mdev = 1.661/3.381/5.101/1.720 ms
mininet> h3 ping h4
PING 10.0.0.4 (10.0.0.4) 56(84) bytes of data.
64 bytes from 10.0.0.4: icmp_seq=1 ttl=64 time=2.40 ms
64 bytes from 10.0.0.4: icmp_seq=2 ttl=64 time=1.14 ms
^C
--- 10.0.0.4 ping statistics ---
2 packets transmitted, 2 received, 0% packet loss, time 1002ms
rtt min/avg/max/mdev = 1.136/1.766/2.396/0.630 ms
mininet> 
\end{lstlisting}
\end{latin}
\item 
\begin{itemize}
	\item \code{\ReverseWords{pingall}}: 
	از هر میزبان به میزبان های دیگر پینگ می‌گیرد.
	\begin{latin}
		\begin{lstlisting}
mininet> pingall
*** Ping: testing ping reachability
h1 -> h2 h3 h4 
h2 -> h1 h3 h4 
h3 -> h1 h2 h4 
h4 -> h1 h2 h3 
*** Results: 0% dropped (12/12 received)
\end{lstlisting}
	\end{latin}
	\item \code{\ReverseWords{nodes}}:
	اجزای شبکه را نشان میدهد. 
	\begin{latin}
\begin{lstlisting}
mininet> nodes
available nodes are: 
c0 h1 h2 h3 h4 s1 s2 s3
\end{lstlisting}
	\end{latin}
	\begin{itemize}
		\item \code{c0}: 
		کنترلر خارجی 
		\item \code{sx}:
		سوئیچ 
		\item \code{hx}: 
		میزبان 
	\end{itemize}
\end{itemize}
\item \code{dump}: 
‫این دستور‬ ‫اطلاعات‬ ‫دقیق‬ ‫و‬ ‫کاملی‬ ‫در‬ ‫مورد‬ ‫هر‬ ‫نود‬ ‫شبکه‬ ‫به‬ ‫ما‬ ‫میدهد‪.‬‬ ‫از‬ ‫جمله‬ ‫آن‬ ‫می‬ ‫توان به‬ ‫نام‪،‬‬ ‫نوع‪،‬‬ ‫اینترفیس‪،‬‬
‫آدرس‬‫‪IP‬‬ ‫‪،‬‬ ‫شماره‬ ‫پروسس‬ ‫اشاره‬ ‫کرد‪.‬‬ ‫در‬ ‫این‬ ‫خروجی‬ ‫سوئیچ‬ ‫به‬ ‫دلیل‬ ‫آنکه‬ ‫آدرس‬ ‫آیپی‬ ‫ندارد‪،‬‬ ‫به‬ ‫ازای‬ ‫همه اینترفیس ها بجز lo 
‫‪None‬‬‫ آمده‬ ‫است‪.‬‬ ‫همچنین‬ ‫نوع‬ ‫سوئیچ‬ ‫‪OVSSwitch‬‬ 
‫می‬ ‫باشد‪.‬‬

\begin{latin}
\begin{lstlisting}
mininet> dump
<Host h1: h1-eth0:10.0.0.1 pid=5400> 
<Host h2: h2-eth0:10.0.0.2 pid=5402> 
<Host h3: h3-eth0:10.0.0.3 pid=5404> 
<Host h4: h4-eth0:10.0.0.4 pid=5406> 
<OVSSwitch s1: lo:127.0.0.1,s1-eth1:None,s1-eth2:None pid=5411> 
<OVSSwitch s2: lo:127.0.0.1,s2-eth1:None,s2-eth2:None,s2-eth3:None pid=5414> 
<OVSSwitch s3: lo:127.0.0.1,s3-eth1:None,s3-eth2:None,s3-eth3:None pid=5417> 
<Controller c0: 127.0.0.1:6653 pid=5393>
\end{lstlisting}
\end{latin}
\end{enumerate}
\end{document}