\documentclass{article}
\usepackage{listings}
\usepackage{xcolor}
\usepackage{enumitem}
\usepackage{graphicx}
\usepackage{amssymb}
\usepackage{bytefield}
\usepackage{forest}
\usepackage{float}
\usepackage{fancyhdr} % Custom headers/footers
\usepackage{colortbl}
\usepackage[left=0.6in, right=0.6in, top=1in, bottom=0.9in]{geometry}
\usepackage{indentfirst}
\usepackage{changepage, titlesec}
\usepackage{booktabs}
\usepackage{array}
\usepackage{listings}
\usepackage{minted}
\usepackage{xcolor}

\usepackage{hyperref}
\usepackage{adjustbox} % for adjustwidth
\usepackage{multicol} % for side-by-side columns
\setlength{\parindent}{1.5em} % Set indentation size (optional)
\usepackage{amsmath} % Required for align environment
\usepackage{xepersian}
\settextfont{Vazirmatn FD}
\setlatintextfont{Noto Serif} 



\lstset{backgroundcolor=\color{gray!20}, basicstyle=\ttfamily}
\newcommand{\code}[1]{\colorbox{gray!10}{\texttt{#1}}}

\pagestyle{fancy}     % Enable fancy headers
\fancyhf{}            % Clear default header/footer
\renewcommand{\headrulewidth}{0pt} % Disable default header line

\fancyhead[L]{\rule{\textwidth}{1pt}} % Manually add one line
\fancyfoot[C]{\thepage} % Page number in the center of the footer

\newcommand{\colorbitbox}[3]{%
	\rlap{\bitbox{#2}{\color{#1}\rule{\width}{\height}}}%
	\bitbox{#2}{#3}}
\definecolor{lightcyan}{rgb}{0.84,1,1}
\definecolor{lightgreen}{rgb}{0.64,1,0.71}
\definecolor{lightred}{rgb}{1,0.7,0.71}

\definecolor{codegreen}{rgb}{0,0.6,0}
\definecolor{codegray}{rgb}{0.5,0.5,0.5}
\definecolor{codepurple}{rgb}{0.58,0,0.82}
\definecolor{backcolour}{rgb}{0.95,0.95,0.92}

\lstdefinestyle{mystyle}{
	backgroundcolor=\color{backcolour},   
	commentstyle=\color{codegreen},
	keywordstyle=\color{magenta},
	numberstyle=\tiny\color{codegray},
	stringstyle=\color{codepurple},
	basicstyle=\ttfamily\footnotesize,
	breakatwhitespace=false,         
	breaklines=true,                 
	captionpos=b,                    
	keepspaces=true,                 
	numbers=left,                    
	numbersep=5pt,                  
	showspaces=false,                
	showstringspaces=false,
	showtabs=false,                  
	tabsize=2
}

\lstset{style=mystyle}

\ExplSyntaxOn
\NewDocumentCommand{\ReverseWords}{m}
{
	\seq_set_split:Nnn \l_tmpa_seq { ~ } { #1 } % Split words by spaces
	\seq_reverse:N \l_tmpa_seq % Reverse the order of words
	\seq_use:Nn \l_tmpa_seq { ~ } % Join words with spaces and output
}
\ExplSyntaxOff


\begin{document}
	\author{ مجتبی ملائی \\ ۴۰۱۳۱۳۸۳ 		}
	\title{ \huge { پروژه دوم}}
	\date{}
	\maketitle
\section{تحلیل ترافیک شبکه}

در این سوال از توپولوژی signel با ۳ میزبان استفاده شد و سویچ به یک کنترل خارجی با پروتکل OpenFlow1.3 متصل گشت. بسته های میزبان اول و همچنین OpenFlow ذخیره شده اند. پس از اجرا، میزبان \lr{h1} از میزبان \lr{h2} پینگ می‌گیرد.
 
\noindent
در میزبان اول بسته های زیر دیده می‌شوند:
\begin{itemize}
	\item \lr{ICMPv6 Router Solicitation}: بسته هایی هستند که میزبان ها برای پیدا کردن روتر در پروتکل \lr{IPv6} ارسال می‌کنند. (در این سوال کابربردی ندارند.) 
	\item :ARP برای پیدا کردن mac-address متناظر با یک IP در یک شبکه محلی
	\item :ICMP برای پینگ گرفتن
\end{itemize}
\noindent
در کنترل خارجی بسته های زیر دیده می‌شود:
\begin{itemize}
	\item \lr{TCP}: ارتباط قابل اطمینان بین کنترلر و سوئیچ
	\item \lr{OpenFlow}: ارتباط و انتقال دستورات و اطلاعات میان کنترلر و سوئیچ
\end{itemize}

\begin{enumerate}[label=\textbf{\Alph*)}]
	\item 
	در پروتکل OpenFlow، هدف از تبادل پیام‌های \ReverseWords{Feature Request} و \ReverseWords{Feature Reply} بین سوئیچ و کنترلر، شناسایی قابلیت‌ها و ویژگی‌های سوئیچ توسط کنترلر است. این تبادل به کنترلر اجازه می‌دهد تا از امکانات سخت‌افزاری و نرم‌افزاری سوئیچ آگاه شود و بر اساس آن، تصمیم‌گیری‌های مناسب انجام دهد.برخی از این اطلاعات شامل: 
	\begin{itemize}
		\item شناسه سوئیچ \ReverseWords{(Datapath ID)}
		\item 
تعداد جدول‌ها \ReverseWords{:(nـtables)} تعداد جدول‌هایی که سوئیچ پشتیبانی می‌کند
		\item 
		قابلیت‌ها 
	\end{itemize}
	\begin{figure}[H]
		\centering
		\includegraphics[width=0.7\linewidth]{screenshot001}
		%\caption{بسته های بین }
		\label{fig:screenshot001}
	\end{figure}
	\begin{figure}[H]
		\centering
		\includegraphics[width=0.5\linewidth]{screenshot002}
		\label{fig:screenshot002}
	\end{figure}
	
	
	\item در این دو سناریو بسته به کنترلر ارسال می‌شود.
	\begin{itemize}
		\item \lr{:\texttt{OFPR\_NO\_MATCH(0)}} وقتی هیچ Flow ای برای بسته در جدول وجود نداشته باشد و اصطلاحا miss رخ دهد.
		\item  \lr{:\texttt{OFPR\_ACTION(1)}} وقتی در Flow متناظر با آن بسته action آن ارسال به کنترلر باشد.
	\end{itemize}
	\item در ارسال بسته های ICMP از پروتکل IP استفاده می‌شود. ابتدا با استفاده از ARP به mac-address مورد نظر دست پیدا می‌کند. سپس بسته های ICMP در بسته های IP  و سپس در فریم Ethernet قرار می‌گیرند. 
	در این پروتکل مقدار type در بسته ها ICMP برابر با 8 قرار می‌گیرد که به معنی \ReverseWords{Echo Request} است. و در پاسخ این فیلد مقدار \lr{0} می‌گیرد که به معنای \ReverseWords{Echo Reply} است. از فیلد ident برای متمایز کردن پینگ بین پروسس های متفاوت استفاده می‌شود و از seq برای ترتیب و متمایز کردن بسته ها استفاده می‌شود. همچنین هر بسته دارای timestamp می‌باشد.
	\begin{figure}[H]
		\centering
		\includegraphics[width=0.7\linewidth]{screenshot003}
		\label{fig:screenshot003}
	\end{figure}
	\begin{figure}[H]
		\centering
		\includegraphics[width=0.5\linewidth]{screenshot004}
		\label{fig:screenshot004}
	\end{figure}
	
\end{enumerate}

\section{‫اجرای‬ ‫توپولوژی‬ ‫های‬‫مختلف‬ ‫در‬ ‫‪Mininet‬‬}

\begin{enumerate}
	\item \code{\ReverseWords{sudo mn --topo single,5 --mac --switch ovsk}}:
	در این دستور از توپولوژی سینگل (یک سوئیچ مرکزی) و ۵ میزبان که به آن متصل هستند استفاده می‌کند. 
	mac-address ها به صورت اتوماتیک تنظیم می‌شوند و از سويچ مجازی در کرنل برای سویچ استفاده می‌شود. 
	\item \code{\ReverseWords{‫‪sudo mn --topo‬‬ single,5 --controller remote -x}}
	در این دستور از توپولوژی سینگل (یک سوئیچ مرکزی) و ۵ میزبان که به آن متصل هستند استفاده می‌کند. آپشن کنترلر باعث می‌شود تا سویچ مرکزی به یک کنترلر خارجی SDN بر روی پورت ۶۶۵۳ طبق پروتکل OpenFlow متصل شود. اپشن x نیز به ازای هر دستگاه (میزبان، سوئیچ، کنترلر) یک shell ایجاد می‌کند. 
	
\end{enumerate}

\end{document}