\documentclass{article}
\usepackage{listings}
\usepackage{xcolor}
\usepackage{enumitem}
\usepackage{graphicx}
\usepackage{amssymb}
\usepackage{bytefield}
\usepackage{forest}
\usepackage{float}
\usepackage{fancyhdr} % Custom headers/footers
\usepackage{colortbl}
\usepackage[left=0.6in, right=0.6in, top=1in, bottom=0.9in]{geometry}
\usepackage{indentfirst}
\usepackage{changepage, titlesec}
\usepackage{booktabs}
\usepackage{array}
\usepackage{hyperref}
\hypersetup{
	colorlinks=true,
	linkcolor=blue,
	urlcolor=blue,
	citecolor=blue,
}
\usepackage{adjustbox} % for adjustwidth
\usepackage{multicol} % for side-by-side columns
\setlength{\parindent}{1.5em} % Set indentation size (optional)
\usepackage{amsmath} % Required for align environment
\usepackage{xepersian}
\settextfont{Vazirmatn FD}
\setlatintextfont{Noto Serif} 


\lstset{backgroundcolor=\color{gray!20}, basicstyle=\ttfamily}
\newcommand{\code}[1]{\colorbox{gray!10}{\texttt{#1}}}


\pagestyle{fancy}     % Enable fancy headers
\fancyhf{}            % Clear default header/footer
\renewcommand{\headrulewidth}{0pt} % Disable default header line

\fancyhead[L]{\rule{\textwidth}{1pt}} % Manually add one line
\fancyfoot[C]{\thepage} % Page number in the center of the footer

\newcommand{\colorbitbox}[3]{%
	\rlap{\bitbox{#2}{\color{#1}\rule{\width}{\height}}}%
	\bitbox{#2}{#3}}
\definecolor{lightcyan}{rgb}{0.84,1,1}
\definecolor{lightgreen}{rgb}{0.64,1,0.71}
\definecolor{lightred}{rgb}{1,0.7,0.71}

\definecolor{codegreen}{rgb}{0,0.6,0}
\definecolor{codegray}{rgb}{0.5,0.5,0.5}
\definecolor{codepurple}{rgb}{0.58,0,0.82}
\definecolor{backcolour}{rgb}{0.95,0.95,0.92}

\lstdefinestyle{mystyle}{
	backgroundcolor=\color{backcolour},   
	commentstyle=\color{codegreen},
	keywordstyle=\color{magenta},
	numberstyle=\tiny\color{codegray},
	stringstyle=\color{codepurple},
	basicstyle=\ttfamily\footnotesize,
	breakatwhitespace=false,         
	breaklines=true,                 
	captionpos=b,                    
	keepspaces=true,                 
	numbers=left,                    
	numbersep=5pt,                  
	showspaces=false,                
	showstringspaces=false,
	showtabs=false,                  
	tabsize=2
}

\lstset{style=mystyle}

\ExplSyntaxOn
\NewDocumentCommand{\ReverseWords}{m}
{
	\seq_set_split:Nnn \l_tmpa_seq { ~ } { #1 } % Split words by spaces
	\seq_reverse:N \l_tmpa_seq % Reverse the order of words
	\seq_use:Nn \l_tmpa_seq { ~ } % Join words with spaces and output
}
\ExplSyntaxOff


\begin{document}
	\author{ مجتبی ملائی \\ ۴۰۱۳۱۳۸۳ }
	\title{ \huge {پروژه سوم}}
	\date{}
	\maketitle
	
\begin{enumerate}
	\item[\textbf{الف)}] 
	تصویر شماتیک از توپولوژی. در این توپولوژی سوئیچ ها در حالت \code{failmode="standalone"} و \code{stp=True} هستند. 
	
	\href{run:topo.py}{\code{topo.py}} 
	\begin{figure}[H]
		\centering
		\includegraphics[width=0.6\linewidth]{screenshot003}
		\label{fig:screenshot003}
	\end{figure}
	
	\item[\textbf{ب)}] 
	نتیجه خروجی دستور \code{net} و دستور \code{pingall}:
\begin{figure}[H]
	\centering
	\label{fig:screenshot002}
	\includegraphics[width=0.65\linewidth]{screenshot002}
		\caption{با توجه پروتکل STP کمی طول می‌کشد تا شبکه کاملا پایدار شود }
\end{figure}
\item[\textbf{پ)}]
جدول زیر مقدار میانگین تاخیر برای مسیر های‌مختلف نشان می‌دهد.
\begin{latin}
\begin{tabular}{|c|c|}
	\hline
	route &  time (ms) \\
	\hline
	h1-s1-h2 & 0.152 \\
	\hline
	h1-s1-s2-h3 & 0.205 \\
	\hline
	h1-s1-s2-h4 & 0.247 \\
	\hline
	h2-s1-s2-h3 & 0.196 \\
	\hline 
	h2-s1-s2-h4 & 0.224 \\
	\hline 
	h3-s2-h4 & 0.146 \\
	\hline
\end{tabular}
\end{latin}


\item[\textbf{ت)}]
جدول زیر مقدار پهنای باند (Bandwidth) بین میزبان‌های مختلف نشان می‌دهد.
\begin{latin}
\begin{tabular}{|c|c|c|}
	\hline
	src & dst &  bandwidth (Gbits/sec) \\
	\hline
	h1& h2 & 43.6 \\
	\hline
	h1 & h3 & 41.8 \\
	\hline
	h1 & h4 & 41.4 \\
	\hline
	h2 & h3 & 41.7 \\
	\hline 
	h2 & h4 & 41.2\\
	\hline
	h3 & h4 & 44.1 \\
	\hline
\end{tabular}
\end{latin}

\item[\textbf{ث)}]
با استفاده از دستور \code{\ReverseWords{py s1.intf('s1-eth1').ifconfig('down')}} اینترفیس اول \lr{s1} را خاموش می‌کند که معادل قطع سیم اول است. (دستور \code{\ReverseWords{link s1 s2 down}} هر دو لینک را قطع می‌کند.) 
\begin{figure}[H]
	\centering
	\includegraphics[width=0.3\linewidth]{screenshot004}
	\caption{شبکه بعد از قطع سیم اول}
	\label{fig:screenshot004}
\end{figure}
پس از گذشت مدتی زمان، همانطور که دیده می‌شود شبکه همچنان به طور کامل کار می‌کند.
\begin{figure}[H]
	\centering
	\includegraphics[width=0.3\linewidth]{screenshot005}
	\caption{شبکه پس از قطع هر دو سیم}
	\label{fig:screenshot005}
\end{figure}
در این حالت، فقط میزبان هایی که به یک سوئیچ وصل هستند کار می‌کنند.
\begin{figure}[H]
	\centering
	\includegraphics[width=0.3\linewidth]{screenshot006}
	\caption{شبکه بعد از وصل مجدد سیم اول}
	\label{fig:screenshot006}
\end{figure}
پس از وصل مجدد سیم اول و با گذشت کمی زمان دوباره شبکه متصل می‌شود. با توجه به خروجی‌ها و رفتار شبکه مشخص است که طبق پروتکل STP مسیر‌ دیگر به درستی جایگزین می‌شود. 
\item[\textbf{ج)}]
\begin{enumerate}[label=\textbf{\arabic*)}]
	\item 

برای چنین کاری، مجبور هستیم حالت STP و standalone را غیر فعال کنیم.  چون STP یکی از پورت‌ها را در حالت \code{STP\_BLOCK} قرار می‌دهد و حتی با تعریف جریان بر روی این پورت، این پورت همچنان بسته‌ها را drop خواهد کرد. بنابراین برای این قسمت از فایل \href{run:topo_no_stp.py}{\code{topo\_no\_stp.py}} استفاده شده است.  


\item 
حال باید یکسری جریان برای کارکرد اولیه شبکه به آن اضافه کنیم.
\begin{latin}
\begin{lstlisting}[language=bash,numbers=none]
sh ovs-ofctl add-flow s1 "priority=70,in_port=2,dl_dst=ff:ff:ff:ff:ff:ff,actions=output:3,4"
sh ovs-ofctl add-flow s2 "priority=70,in_port=2,dl_dst=ff:ff:ff:ff:ff:ff,actions=output:3,4"
sh ovs-ofctl add-flow s1 "priority=70,in_port=3,dl_dst=ff:ff:ff:ff:ff:ff,actions=output:2,4"
sh ovs-ofctl add-flow s1 "priority=70,in_port=4,dl_dst=ff:ff:ff:ff:ff:ff,actions=output:3,2"
sh ovs-ofctl add-flow s2 "priority=70,in_port=3,dl_dst=ff:ff:ff:ff:ff:ff,actions=output:2,4"
sh ovs-ofctl add-flow s2 "priority=70,in_port=4,dl_dst=ff:ff:ff:ff:ff:ff,actions=output:3,2"
	\end{lstlisting}
\end{latin}

این جریان‌ها برای مدیریت broadcast ها در درخواست های arp در نظر گرفته شده است و به نحوی است که حلقه ایجاد نشود. از لینک دوم بین دو سوئیچ برای  انتقال این بسته ها استفاده شده است. 
\item 
برای مسیر های عادی بین میزبان‌ها نیز جریان‌های زیر را داریم:
\begin{latin}
\begin{lstlisting}[language=bash,numbers=none]
sh ovs-ofctl add-flow s1 "priority=75,dl_dst=00:00:00:00:00:01,actions=output:3"
sh ovs-ofctl add-flow s1 "priority=75,dl_dst=00:00:00:00:00:02,actions=output:4"
sh ovs-ofctl add-flow s2 "priority=75,dl_dst=00:00:00:00:00:03,actions=output:3"
sh ovs-ofctl add-flow s2 "priority=75,dl_dst=00:00:00:00:00:04,actions=output:4"
sh ovs-ofctl add-flow s2 "priority=75,dl_dst=00:00:00:00:00:01,actions=output:2"
sh ovs-ofctl add-flow s2 "priority=75,dl_dst=00:00:00:00:00:02,actions=output:2"
sh ovs-ofctl add-flow s1 "priority=75,dl_dst=00:00:00:00:00:03,actions=output:2"
sh ovs-ofctl add-flow s1 "priority=75,dl_dst=00:00:00:00:00:04,actions=output:2"
\end{lstlisting}
\end{latin}  
\item 
نهایتا برای هدایت ترافیک بین دو میزبان (مثلا \lr{h1} و \lr{h4}) می‌توانیم جریان هایی به این شکل بنویسیم:
\begin{latin}
\begin{lstlisting}[language=bash,numbers=none]
sh ovs-ofctl add-flow s1 "priority=80,ip,nw_src=10.0.0.1,nw_dst=10.0.0.4,actions=output:1"
sh ovs-ofctl add-flow s2 "priority=80,ip,nw_src=10.0.0.4,nw_dst=10.0.0.1,actions=output:1"
\end{lstlisting}
\end{latin}
در این حالت این جریان ها به دلیل اولویت بیشترشان اول بررسی می‌شوند. این جریان‌ها ترافیک بین \lr{h1} و \lr{h4} را از طرق لینک اول عبور می‌دهند. درحالی که بقیه ترافیک ها از سیم دوم عبور می‌کنند.
\item 
برای بررسی صحت این موضوع از پینگ و \code{dump-flows} استفاده می‌کنیم.

\begin{figure}[H]
	\centering
	\label{fig:screenshot011}
	\includegraphics[width=0.7\linewidth]{screenshot011}
\end{figure}
همانطور که دیده می‌شود، بسته های پینگ از جریانی که از لینک اول عبور می‌کند، استفاده می‌کند.
\end{enumerate}
\vspace{4cm}

\item[\textbf{چ)}]
برای اینکار، از یک سوئیچ سوم استفاده می‌کنیم. و سوئیچ دوم و اول را به آن وصل می‌کنیم. اگر این سوئیچ یا یکی از لینک های متصل به آن دچار خطا شود، ارتباط بین نیمه سمت چپ و راست قطع می‌شود. \href{run:singel_point_of_failure.py}{\code{singel\_point\_of\_failure.py}}
\begin{figure}[H]
	\centering
	\includegraphics[width=0.7\linewidth]{screenshot012}
	\label{fig:screenshot012}
\end{figure}
 برای امتحان کردن این موضوع لینک بین \lr{s1} و \lr{s3} را قطع کرده و \code{pingall} را اجرا می‌کنیم.
 \begin{figure}[H]
 	\centering
 	\includegraphics[width=0.35\linewidth]{screenshot013}
 	\label{fig:screenshot013}
 \end{figure}
 
 	برای حل این مشکل کافی است یک لینک بین \lr{s1} و \lr{s2} ایجاد کنیم. حال دوباره آن را با \code{pingall} تست می‌کنیم.
 	
 	\href{run:single_point_of_failure_redundancy.py}{\code{single\_point\_of\_failure\_redundancy.py}}
 
 \begin{figure}[H]
 	\centering
 	\includegraphics[width=0.35\linewidth]{screenshot014}
 	\label{fig:screenshot014}
 \end{figure}
 همانطور که دیده می‌شود. شبکه بدون مشکل به کار خود ادامه می‌دهد چون از سیم اضافی که بین سوئیچ یک و دو وجود دارد استفاده کرده است. 
\end{enumerate}
\end{document}