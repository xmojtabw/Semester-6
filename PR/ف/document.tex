%% NOTICE: Overleaf free plan is limiting this compilation, so it is better to compile using your own computer
\documentclass{beamer}

%% TODO: Uncomment the next three lines to add notes to presentation slides
%\usepackage{pgfpages}
%\setbeameroption{show notes on second screen}
%\setbeamertemplate{note page}{\insertnote}

%\usepackage[utf8]{inputenc}
\usepackage{fontawesome5}

\usepackage{fontspec}

\usepackage{graphicx}

%% TODO: Put all the figure files inside the images folder
\graphicspath{{images/}}

\usepackage{amsmath}
\usepackage{amssymb}
\usepackage{caption}
\usepackage{pifont}

%% TODO: Use \cmark for tick and \xmark for x
\newcommand{\cmark}{\ding{51}}%
\newcommand{\xmark}{\ding{55}}%

\usepackage{algorithm}
\usepackage{algpseudocode}

%% TODO: Comment the next three lines to remove the bibliography
\usepackage[backend=biber,style=numeric, citestyle=ieee]{biblatex}
\addbibresource{bibliography.bib}
\AtBeginBibliography{\small}

\usepackage{appendixnumberbeamer}
\pdfstringdefDisableCommands{%
	\def\translate#1{#1}%
}

\usepackage{xepersian}
\settextfont{Vazirmatn FD}
\setlatintextfont{Noto Serif} 

%% TODO: Comment the next line if you need the control symbol
\beamertemplatenavigationsymbolsempty

\usetheme{Berlin}
\useinnertheme{circles}

\makeatletter
\def\beamer@@frametitle[#1]#2{%
	\gdef\insertframetitle{\centering{#2\ifnum\beamer@autobreakcount>0\relax{}\space\usebeamertemplate*{frametitle continuation}\fi}}%
	\gdef\beamer@frametitle{#2}%
	\gdef\beamer@shortframetitle{#1}}
\makeatother


\ExplSyntaxOn
\NewDocumentCommand{\ReverseWords}{m}
{
	\seq_set_split:Nnn \l_tmpa_seq { ~ } { #1 } % Split words by spaces
	\seq_reverse:N \l_tmpa_seq % Reverse the order of words
	\seq_use:Nn \l_tmpa_seq { ~ } % Join words with spaces and output
}
\ExplSyntaxOff


\AtBeginSection[]
{
	\begin{frame}
		\frametitle{فهرست مطالب}
		\tableofcontents[currentsection]
	\end{frame}
}

%% TODO: Add information to the title page
\title{کاربرد های نوین شبکه های عصبی و یادگیری عمیق}
\author{کوروش جمشیدی \and سامان اصغری \and مجتبی ملائی \and مهتا میرزائی}
\institute{
	دانشکده مهندسی برق و کامیپوتر \\
	دانشگاه صنعتی اصفهان	
}
\date[short date]{فروردین ۱۴۰۴}
%\logo{\includegraphics[width=.05\linewidth]{IU}}

%% Page number
\expandafter\def\expandafter\insertshorttitle\expandafter{%
	\insertshorttitle\hfill%
	\insertframenumber\,/\,\inserttotalframenumber}

\setbeamertemplate{footline}{
	\leavevmode%
	\hbox{%
		\begin{beamercolorbox}[wd=.15\paperwidth,ht=2.5ex,dp=1ex,right]{date in head/foot}%
			\centering
			\insertframenumber/\inserttotalframenumber\hspace*{2ex}
		\end{beamercolorbox}%
		\begin{beamercolorbox}[wd=.7\paperwidth,ht=2.5ex,dp=1ex,center]{title in head/foot}%
			دانشگاه صنعتی اصفهان - دانشکده مهندسی کامپیوتر
		\end{beamercolorbox}%
		\begin{beamercolorbox}[wd=.15\paperwidth,ht=2.5ex,dp=1ex,left]{author in head/foot}%
			\raisebox{0.3cm}{\includegraphics[width=0.8cm]{IUT}}%
			\hspace{0.2cm}
		\end{beamercolorbox}%
		
	}%
	\vskip0pt%
}



\begin{document}
	
	%% Title frame
	\begin{frame}
		\titlepage
		%% TODO: You can add the note here
		\note{}
	\end{frame}
	
	%% ToC frame
	\begin{frame}{فهرست مطالب}
		\tableofcontents
		%% TODO: You can add the note here
		\note{}
	\end{frame}
	
	\section{معرفی} 
	
	%% TODO: Make sure to use \alert{} for highlighting keywords, and \cite{} to cite the corresponding quotations
	\begin{frame}{انگیزه ها}
		\begin{itemize} 	
			\item 
			این اولین \alert{کلمه کلیدی هایلایت شده} برای امتحان کردن است.
			\item 
			برای دومین مثال از \alert{دومین کلمه کلیدی} به همراه رفرنس در \cite{knuth:1984}.
			
		\end{itemize}
		%% TODO: You can add the note here
		\note{}
	\end{frame}
	
	%% TODO: If you need a button, label the frames accordingly then using \hyperlink{label_of_the_dest_frame}{\beamerbutton{Name of the dest. frame}
		\begin{latin}
			\begin{frame}[label=objectives]{هدف‌ها \hyperlink{scope}{\beamerbutton{Scope}}}
				
				
				
				\begin{block}{Sample Block Title}
					This block presents a \alert{key concept} that is crucial for understanding the topic.
				\end{block}
				\begin{alertblock}{Sample Alert Block Title}
					This block presents a more alarming \alert{key concept} that is crucial for understanding the topic.
				\end{alertblock}
				%% TODO: You can add the note here
				\note{}
				
			\end{frame}
		\end{latin}
		\begin{frame}{Actors \& Features}
			\textbf{Actors:}
			
			\textbf{Features:}
			%% TODO: You can add the note here
			\note{}
		\end{frame}
		
		\begin{frame}{Contributions}				
			\begin{block}{Scientific Contribution}
			\end{block}						
			\begin{block}{Real-world Contribution}
			\end{block}					
			%% TODO: You can add the note here
			\note{}
		\end{frame}
		
		\section{یک مدل CNN برای تشخیص ضربان قلب}
		
		%% TODO: Uncomment the next four lines to enable this frame
		% \begin{frame}{Advancements}
			% 	%% TODO: You can add the note here
			% 	\note{}
			% \end{frame}
		
		\begin{frame}{ECG چیست؟}
			\begin{itemize}
				\item ثبت فعالیت قلب 
				\item رسم سیگنال در قالب نمودار
				\item یک تست ساده و غیر قابل تهاجمی
				\item هر ضربان قلب نمایانگر یک سیگنال
				\item قابل استفاده در تشخیص بیماری
				
			\end{itemize}
			
			%% TODO: You can add the note here
			\note{}
		\end{frame}
		
		\begin{frame}{آریتمی چیست؟)}
			\begin{itemize}
				\item ریتم غیرطبیعی قلب
				\item سریع یا کند بودن ضربان قلب
				\item نامنظم بودن ضربان قلب
				\item ECG ابزار تشخیص بیماری آریتمی
			\end{itemize}
		\end{frame}
		
		\begin{frame}{وجود شبکه عصبی}
			\begin{itemize}
				\item تفسیر دستی ECG زمان‌بر و دشوار
				\item نیاز به سیستم‌های خودکار
				\item افزایش دقت و سرعت با استفاده از سیستم های هوشمند
				
			\end{itemize}
			
		\end{frame}
		
		\begin{frame}{دسته بندی انواع ضربان ها}
			\begin{itemize}
				\item[N] $\leftarrow$ ضربان‌های نرمال
				\item[S] $\leftarrow$ ضربان‌های فوق‌بطنی نابجا
				\item[V] $\leftarrow$ ضربان‌های بطنی نابجا
				\item[F] $\leftarrow$ ضربان‌های ترکیبی (فیوژن)
				\item[Q] $\leftarrow$ ضربان‌های ناشناخته یا غیرقابل طبقه‌بندی
			\end{itemize}
		\end{frame}
		
		\begin{frame}{چالش داده ها}
			\begin{itemize}
				\item تکراری بودن دسته N
				\item کمبود داده ها در سایر دسته‌ها
				\item عدم دقت مدل به علت ناتوازنی داده‌ها
				\item تولید داده مصنوعی برای تعادل
				\item افزایش کلاس‌های کم‌نمونه به اندازه N
				
			\end{itemize}
		\end{frame}
		
		
		\begin{frame}{معماری شبکه}
			\footnotesize
			\centering
			\captionsetup{skip=0pt}
			\begin{figure}
				\includegraphics[width=0.8\linewidth,height=0.3\linewidth]{images/ecg1}
				\caption*{\scriptsize شکل ۱: ساخته‌شده توسط ChatGPT}
			\end{figure}
			
			%\vspace{0.5em}
			\begin{columns}
				\column{0.5\textwidth}
				\begin{itemize}
					\item ۳ لایه کانولوشن برای استخراج ویژگی
					\item ۳ لایه \ReverseWords{max pooling} برای کاهش ابعاد 
					\item ۳ لایه \ReverseWords{fully connected} نهایی
				\end{itemize}
				
				\column{0.5\textwidth}
				\begin{itemize}
					\item خروجی softmax با ۵ دسته مختلف
					\item فعال‌ساز \ReverseWords{Leaky ReLU} در لایه‌ها
					\item آموزش با backpropagation در ۲۰ تکرار
				\end{itemize}
			\end{columns}
		\end{frame}
		
		
		
		\begin{frame}{نتایج}
			\begin{itemize}
				\item دقت بالا در تشخیص دسته‌ها
				\item عملکرد بهتر با داده تمیز \ReverseWords{(Set B)}
				\item دقت نهایی حدود 94 درصد
				\item حساسیت بالا در تمام دسته‌ها
				\item بهترین عملکرد توسط کلاس Q
				
			\end{itemize}
		\end{frame}
		
		\begin{frame}{مزایا و معایب}
			\begin{columns}
				\column{0.5\textwidth}
				\begin{block}{مزایا}
					\begin{itemize}
						\item[\faCheck] کاملاً خودکار و سریع
						\item[\faCheck] مقاوم در برابر نویز سیگنال
						\item[\faCheck] بدون نیاز به استخراج ویژگی
					\end{itemize}
				\end{block}
				
				
				
				\column{0.5\textwidth}
				\begin{alertblock}{معایب}
					\begin{itemize}
						\item[\faTimes] آموزش سنگین و زمان‌بر
						\item[\faTimes] نیاز به سخت‌افزار قوی 
						\item[\faTimes] احتمال وجود خطا
					\end{itemize}
				\end{alertblock}
				
				
			\end{columns}
		\end{frame}
		
		\begin{frame}{نتیجه گیری}
			\begin{itemize}
				\item[\faClipboardCheck] CNN یک ابزار قدرتمند در پزشکی
				\item[\faClipboardCheck] دقت بالا در طبقه‌بندی ضربان
				\item[\faClipboardCheck] کاهش زمان بررسی ECG
				\item[\faClipboardCheck] قابل اجرا در سیستم‌های بالینی
				\item[\faClipboardCheck] مناسب برای غربال‌گری سریع بیماران
				
			\end{itemize}
		\end{frame}
		\begin{frame}{در آینده}
			\centering
			\LARGE  \faQuestionCircle  \hspace{0.2cm}  تشخیص توالی‌های زمانی  ECG  
		\end{frame}
		
		%% TODO: Uncomment the next four lines to enable this frame
		% \begin{frame}{Technical difficulties}				
			% 	%% TODO: You can add the note here
			% 	\note{}
			% \end{frame}
		
		\section{تشخیص deepfake توسط یادگیری عمیق} 
		
		%% TODO: Adjust the size of the figure by using [scale=x] or [widht=.x\linewidth] (x is a fraction) to fit within the frame. Then rename to your picture file name and add the caption
		\begin{frame}{Overview}
			\begin{figure}
				\centering
				\includegraphics[width=.8\linewidth]{samplel.png}
				\caption{The caption of the figure.}
			\end{figure}	
			%% TODO: You can add the note here
			\note{}
		\end{frame}
		
		%% TODO: Uncomment the next four lines to enable this frame
		% \begin{frame}{Methodology}
			% 	%% TODO: You can add the note here
			% 	\note{}
			% \end{frame}
		
		\begin{frame}[label=process1]{\texttt{Sample} Process \hyperlink{algo1}{\beamerbutton{Algorithm}} \hyperlink{pseudocode1}{\beamerbutton{Pseudocode}}}
			\begin{columns}
				\column{0.3\textwidth}
				\centering
				\includegraphics[height=1.8\textwidth]{samplev.png}
				\column{0.7\textwidth}
				\begin{itemize}
					\item \textbf{Goal:}
					\item \textbf{Result:}
					\item \textbf{Step:} 
					\item \textbf{Scope:}
				\end{itemize}
			\end{columns}
			%% TODO: You can add the note here
			\note{}
		\end{frame}
		
		
		\section{تشخیص فیشینگ توسط CNN} 
		
		\begin{frame}{چرا تشخیص فیشینگ چالش برانگیز است؟
			}
			%% TODO: Add your GitHub repository link here
			\textbf{GitHub repository:} \url{}
			
			%% TODO: Add your demo website link here
			\textbf{Demo Website:} \url{}
			
			\begin{columns}
				\column{0.5\textwidth}
				\begin{figure}
					\centering
					\includegraphics[width=.8\textwidth]{samples1.png}
					\caption{The caption of the figure.}
				\end{figure}	
				\column{0.5\textwidth}
				\begin{figure}
					\centering
					\includegraphics[width=.8\textwidth]{samples2.png}
					\caption{The caption of the figure.}
				\end{figure}
			\end{columns}
			%% TODO: You can add the note here
			\note{}
		\end{frame}
		
		%% TODO: Uncomment the next four lines to enable this frame
		% \begin{frame}{Benchmark Results}
			% 	%% TODO: You can add the note here
			% 	\note{}	
			% \end{frame}
		
		\section{Discussion} 
		
		%% TODO: Uncomment the next four lines to enable this frame
		% \begin{frame}{Analysis}
			% 	%% TODO: You can add the note here
			% 	\note{}
			% \end{frame}
		
		\begin{frame}{Limitations}
			$\Rightarrow$ \textbf{Concluding statement.}
			%% TODO: You can add the note here
			\note{}
		\end{frame}
		
		\begin{frame}{Comparison}
			\begin{table}[ht]
				\centering
				\caption{Comparison of different methods (\protect\cmark: YES, \protect\xmark: NO).}
				%% Comment the next line if the table width is relatively small
				\resizebox{\textwidth}{!}{%
					\begin{tabular}{lcccccc}
						\hline
						& \textbf{Your Method} & Method B & Method C & Method D & Method E & Method F \\ \hline
						Feature 1 & \cmark               & \cmark   & \xmark   & \cmark   & \xmark   & \cmark   \\ 
						Feature 2 & \cmark               & \xmark   & \cmark   & \cmark   & \cmark   & \xmark   \\ 
						Feature 3 & \xmark               & \cmark   & \cmark   & \xmark   & \xmark   & \cmark   \\ 
						Feature 4 & \cmark               & \cmark   & \xmark   & \xmark   & \cmark   & \xmark   \\ 
						Feature 5 & \xmark               & \xmark   & \cmark   & \cmark   & \xmark   & \cmark   \\ 
						Feature 6 & \cmark               & \xmark   & \cmark   & \xmark   & \xmark   & \xmark   \\ \hline
					\end{tabular}%
					%%TODO: Also comment this } to match the above command
			}
		\end{table}
		%% TODO: You can add the note here
		\note{}
	\end{frame}
	
	\section{Conclusion}
	
	%% TODO: Uncomment the next four lines to enable this frame
	% \begin{frame}{Summary}
		% 	%% TODO: You can add the note here
		% 	\note{}
		% \end{frame}
	
	%% TODO: Uncomment the next four lines to enable this frame
	% \begin{frame}{Future works}		
		% 	%% TODO: You can add the note here
		% 	\note{}
		% \end{frame}
	
	\begin{frame}{Demonstration}
		\begin{block}{Process A}
		\end{block}
		\begin{block}{Scenario 1}
		\end{block}
		\begin{alertblock}{Scenario 2}
		\end{alertblock}
		\begin{block}{Process B}
		\end{block}
		%% TODO: You can add the note here
		\note{}
	\end{frame}
	
	%% Thank You frame
	\begin{frame}
		\centering
		\includegraphics[width=.7\linewidth]{thankyou}
		%% TODO: You can add the note here
		\note{}
	\end{frame}
	
	%% Appendix frames
	\appendix
	
	\begin{frame}[label=scope]{Scope \hyperlink{objectives}{\beamerbutton{Back to Objectives}}}
		%% TODO: You can add the note here
		\note{}
	\end{frame}
	
	\begin{frame}[label=algo1]{Formalizing - \texttt{Sample} Algorithm \hyperlink{process1}{\beamerbutton{Back to $\texttt{Sample}$ process}}}
		\begin{algorithm}[H]
			\small
			\caption{$(\text{Result}) \gets \texttt{Sample}(\text{Input1})$}
			\label{alg:algo1}
			\begin{algorithmic}[1]
				\Require $\text{Input1}$ is a predefined parameter.
				\State $\text{Set} \gets \emptyset$
				\For{$\text{element} \in \text{Input1}$}
				\If{$\text{Condition}(\text{element})$ is true}
				\State $\text{Set} \gets \text{Set} \cup \{\text{Process}(\text{element})\}$
				\Else
				\State \textbf{continue}
				\EndIf
				\EndFor
				\State $\text{Intermediate} \gets \texttt{Transform}(\text{Set})$
				\State \Return $\text{Result}$
			\end{algorithmic}
		\end{algorithm}
		%% TODO: You can add the note here
		\note{}
	\end{frame}
	
	\begin{frame}[label=pseudocode1]{Formalizing - \texttt{Sample} Pseudocode \hyperlink{process1}{\beamerbutton{Back to $\texttt{Sample}$ process}}}
		\begin{latin}
			
			\begin{algorithm}[H]
				\small
				\caption{$(\text{Result}) \gets \texttt{Sample}(\text{Input1})$}
				\label{alg:pseudocode1}
				\begin{algorithmic}[1]
					\Require $\text{Input1}$ is a predefined parameter.
					\State $\text{Set} \gets \emptyset$
					\For{$\text{element} \in \text{Input1}$}
					\If{$\text{Condition}(\text{element})$ is true}
					\State $\text{Set} \gets \text{Set} \cup \{\text{Process}(\text{element})\}$
					\Else
					\State \textbf{continue}
					\EndIf
					\EndFor
					\State $\text{Intermediate} \gets \texttt{Transform}(\text{Set})$
					\State \Return $\text{Result}$
				\end{algorithmic}
			\end{algorithm}
		\end{latin}
		%% TODO: You can add the note here
		\note{}
	\end{frame}
	
	%% TODO: Uncomment the next four lines to enable this frame
	% \begin{frame}{System architecture}
		% 	%% TODO: You can add the note here
		% 	\note{}
		% \end{frame}
	
	%% TODO: Uncomment the next four lines to enable this frame
	% \begin{frame}{Tech strategy choices}
		% 	%% TODO: You can add the note here
		% 	\note{}
		% \end{frame}
	
	\begin{frame}[allowframebreaks, noframenumbering]{References}
		\begin{latin}
			\printbibliography[heading=none]
		\end{latin}
		
	\end{frame}
	
\end{document}