%% NOTICE: Overleaf free plan is limiting this compilation, so it is better to compile using your own computer
\documentclass{beamer}

%% TODO: Uncomment the next three lines to add notes to presentation slides
%\usepackage{pgfpages}
%\setbeameroption{show notes on second screen}
%\setbeamertemplate{note page}{\insertnote}

%\usepackage[utf8]{inputenc}
\usepackage{fontawesome5}

\usepackage{fontspec}

\usepackage{graphicx}

%% TODO: Put all the figure files inside the images folder
\graphicspath{{images/}}

\usepackage{amsmath}
\usepackage{amssymb}
\usepackage{caption}
\usepackage{pifont}

%% TODO: Use \cmark for tick and \xmark for x
\newcommand{\cmark}{\ding{51}}%
\newcommand{\xmark}{\ding{55}}%

\usepackage{algorithm}
\usepackage{algpseudocode}

%% TODO: Comment the next three lines to remove the bibliography
\usepackage[backend=biber,style=numeric, citestyle=ieee]{biblatex}
\addbibresource{bibliography.bib}
\AtBeginBibliography{\small}

\usepackage{appendixnumberbeamer}
\pdfstringdefDisableCommands{%
	\def\translate#1{#1}%
}

 \usepackage{xepersian}
 \settextfont{Vazirmatn FD}
 \setlatintextfont{Noto Serif} 

%% TODO: Comment the next line if you need the control symbol
\beamertemplatenavigationsymbolsempty

\usetheme{Berlin}
\useinnertheme{circles}

\makeatletter
\def\beamer@@frametitle[#1]#2{%
	\gdef\insertframetitle{\centering{#2\ifnum\beamer@autobreakcount>0\relax{}\space\usebeamertemplate*{frametitle continuation}\fi}}%
	\gdef\beamer@frametitle{#2}%
	\gdef\beamer@shortframetitle{#1}}
\makeatother


\ExplSyntaxOn
\NewDocumentCommand{\ReverseWords}{m}
{
	\seq_set_split:Nnn \l_tmpa_seq { ~ } { #1 } % Split words by spaces
	\seq_reverse:N \l_tmpa_seq % Reverse the order of words
	\seq_use:Nn \l_tmpa_seq { ~ } % Join words with spaces and output
}
\ExplSyntaxOff


\AtBeginSection[]
{
	\begin{frame}
		\frametitle{فهرست مطالب}
		\tableofcontents[currentsection]
	\end{frame}
}

%% TODO: Add information to the title page
\title{کاربرد های نوین شبکه های عصبی و یادگیری عمیق}
\author{کوروش جمشیدی \\ سامان اصغری \\ مجتبی ملائی \\ مهتا میرزائی}
\institute{
دانشکده مهندسی کامپیوتر \\
دانشگاه صنعتی اصفهان	
  }
\date[short date]{فروردین ۱۴۰۴}
%\logo{\includegraphics[width=.05\linewidth]{IU}}

%% Page number
\expandafter\def\expandafter\insertshorttitle\expandafter{%
	\insertshorttitle\hfill%
	\insertframenumber\,/\,\inserttotalframenumber}
	
\setbeamertemplate{footline}{
	\leavevmode%
	\hbox{%
		\begin{beamercolorbox}[wd=.15\paperwidth,ht=2.5ex,dp=1ex,right]{date in head/foot}%
			\centering
			\insertframenumber/\inserttotalframenumber\hspace*{2ex}
		\end{beamercolorbox}%
		\begin{beamercolorbox}[wd=.7\paperwidth,ht=2.5ex,dp=1ex,center]{title in head/foot}%
			دانشگاه صنعتی اصفهان - دانشکده مهندسی کامپیوتر
		\end{beamercolorbox}%
\begin{beamercolorbox}[wd=.15\paperwidth,ht=2.5ex,dp=1ex,left]{author in head/foot}%
	\raisebox{0.3cm}{\includegraphics[width=0.8cm]{IUT}}%
	\hspace{0.2cm}
\end{beamercolorbox}%

	}%
	\vskip0pt%
}

	

\begin{document}
	
	%% Title frame
	\begin{frame}
		\titlepage
		%% TODO: You can add the note here
		\note{}
	\end{frame}
	
	%% ToC frame
	\begin{frame}{فهرست مطالب}
		\tableofcontents
		%% TODO: You can add the note here
		\note{}
	\end{frame}
	
	\section{معرفی} 
	
	%% TODO: Make sure to use \alert{} for highlighting keywords, and \cite{} to cite the corresponding quotations
	\begin{frame}{شبکه عصبی}
		\begin{columns}
			\column{0.5\textwidth}
			\begin{itemize}
				\item نورون
				\item لایه‌های مختلف
				\item کانال‌ها
				\item Activation Function
				\item Forward Propagation 
				\item تشخیص خطا
				\item Backward Propagation
			\end{itemize}
			\column{0.5\textwidth}
			\begin{figure}
				\centering
				\includegraphics[width=0.97\linewidth]{images/screenshot007}
				\caption*{\scriptsize شکل ۱-۱: شبکه عصبی\\ ساخته شده با ChatGPT}
				\label{fig:screenshot007}
			\end{figure}
		\end{columns}
	\end{frame}
	
\begin{frame}{مثال}
	\begin{figure}
		\centering
		\noindent
		\vspace{-0.6cm}
		\hspace{5cm}
		\includegraphics[width=1\textwidth,height=0.77\textheight]{images/screenshot008}
		\caption*{\scriptsize شکل ۱-۲: یک مثال از شبکه عصبی \lr{https://youtube.com/Simplilearn }}
		\label{fig:screenshot008}
\end{figure}
\end{frame}


		

	\begin{frame}{شبکه های عصبی کانولوشنی (CNN)}
		\begin{figure}
			\centering
			\includegraphics[width=1\textwidth]{images/screenshot009}
			\caption{\scriptsize شکل ۱-۳: شبکه های CNN ساخته شده با ChatGPT}
			\label{fig:screenshot009}
		\end{figure}
		
	\end{frame}
	%% TODO: If you need a button, label the frames accordingly then using \hyperlink{label_of_the_dest_frame}{\beamerbutton{Name of the dest. frame}
		
		\section{یک مدل CNN برای تشخیص ضربان قلب}
		
		%% TODO: Uncomment the next four lines to enable this frame
		% \begin{frame}{Advancements}
			% 	%% TODO: You can add the note here
			% 	\note{}
			% \end{frame}
		
		\begin{frame}{ECG چیست؟}
			\begin{itemize}
				\item ثبت فعالیت قلب 
				\item رسم سیگنال در قالب نمودار
				\item یک تست ساده و غیر قابل تهاجمی
				\item هر ضربان قلب نمایانگر یک سیگنال
				\item قابل استفاده در تشخیص بیماری
				
			\end{itemize}
			
			%% TODO: You can add the note here
			\note{}
		\end{frame}
		
		\begin{frame}{آریتمی چیست؟}
			\begin{itemize}
				\item ریتم غیرطبیعی قلب
				\item سریع یا کند بودن ضربان قلب
				\item نامنظم بودن ضربان قلب
				\item ECG ابزار تشخیص بیماری آریتمی
			\end{itemize}
		\end{frame}
		
		\begin{frame}{وجود شبکه عصبی}
			\begin{itemize}
				\item تفسیر دستی ECG زمان‌بر و دشوار
				\item نیاز به سیستم‌های خودکار
				\item افزایش دقت و سرعت با استفاده از سیستم های هوشمند
				
			\end{itemize}
			
		\end{frame}
		
		\begin{frame}{دسته بندی انواع ضربان‌ها}
			\begin{itemize}
				\item[N] $\leftarrow$ ضربان‌های نرمال
				\item[S] $\leftarrow$ ضربان‌های فوق‌بطنی نابجا
				\item[V] $\leftarrow$ ضربان‌های بطنی نابجا
				\item[F] $\leftarrow$ ضربان‌های ترکیبی (فیوژن)
				\item[Q] $\leftarrow$ ضربان‌های ناشناخته یا غیرقابل طبقه‌بندی
			\end{itemize}
		\end{frame}
		
		\begin{frame}{چالش داده‌ها}
			\begin{itemize}
				\item تکراری بودن دسته N
				\item کمبود داده‌ها در سایر دسته‌ها
				\item عدم دقت مدل به علت ناتوازنی داده‌ها
				\item تولید داده مصنوعی برای تعادل
				\item افزایش کلاس‌های کم‌نمونه به اندازه N
				
			\end{itemize}
		\end{frame}

		
\begin{frame}{معماری شبکه}

\begin{columns}
	\column{0.5\textwidth}
	\begin{itemize}
		\item ۳ لایه کانولوشن برای استخراج ویژگی
		\item ۳ لایه \ReverseWords{max pooling} برای کاهش ابعاد 
		\item فعال‌ساز \ReverseWords{Leaky ReLU} در لایه‌ها
	\end{itemize}
	\column{0.55\textwidth}
	\begin{figure}
		\includegraphics[width=0.93\linewidth,height=0.9\linewidth]{images/ecg11}
		\caption*{\scriptsize شکل ۲-۱: قسمت اول معماری شبکه مورد نظر ساخته‌شده توسط ChatGPT}
	\end{figure}
\end{columns}
\end{frame}

\begin{frame}{معماری شبکه (ادامه)}

	\centering
	\begin{columns}
		\column{0.5\textwidth}
		\begin{itemize}
			\item ۳ لایه \ReverseWords{fully connected} نهایی
			\item خروجی softmax با ۵ دسته مختلف
			\item آموزش با backpropagation در ۲۰ تکرار
		\end{itemize}
		
		\column{0.5\textwidth}
		\begin{figure}
			\includegraphics[width=0.93\linewidth,height=0.9\linewidth]{images/ecg12}
			\caption*{\scriptsize شکل ۲-۲: قسمت دوم معماری شبکه مورد نظر ساخته‌شده توسط ChatGPT}
		\end{figure}
	\end{columns}
\end{frame}



\begin{frame}{نتایج}
	\begin{itemize}
		\item دقت بالا در تشخیص دسته‌ها
		\item عملکرد بهتر با داده تمیز \ReverseWords{(Set B)}
		\item دقت نهایی حدود 94 درصد
		\item حساسیت بالا در تمام دسته‌ها
		\item بهترین عملکرد توسط کلاس Q
		
	\end{itemize}
\end{frame}

\begin{frame}{مزایا و معایب}
	\begin{columns}
		\column{0.5\textwidth}
		\begin{block}{مزایا}
		\begin{itemize}
			\item[\faCheck] کاملاً خودکار و سریع
			\item[\faCheck] مقاوم در برابر نویز سیگنال
			\item[\faCheck] بدون نیاز به استخراج ویژگی
		\end{itemize}
		\end{block}
		

		
		\column{0.5\textwidth}
		 \begin{alertblock}{معایب}
		 \begin{itemize}
		 	\item[\faTimes] آموزش سنگین و زمان‌بر
		 	\item[\faTimes] نیاز به سخت‌افزار قوی 
		 	\item[\faTimes] احتمال وجود خطا
		 \end{itemize}
		 \end{alertblock}

		
	\end{columns}
\end{frame}

\begin{frame}{نتیجه گیری}
	\begin{itemize}
		\item[\faClipboardCheck] CNN یک ابزار قدرتمند در پزشکی
		\item[\faClipboardCheck] دقت بالا در طبقه‌بندی ضربان
		\item[\faClipboardCheck] کاهش زمان بررسی ECG
		\item[\faClipboardCheck] قابل اجرا در سیستم‌های بالینی
		\item[\faClipboardCheck] مناسب برای غربال‌گری سریع بیماران
		
	\end{itemize}
\end{frame}
%\begin{frame}{در آینده}
%	\centering
%	\LARGE  \faQuestionCircle  \hspace{0.2cm}  تشخیص توالی‌های زمانی  ECG  
%\end{frame}

		%% TODO: Uncomment the next four lines to enable this frame
		% \begin{frame}{Technical difficulties}				
			% 	%% TODO: You can add the note here
			% 	\note{}
			% \end{frame}
		
\section{تشخیص deepfake توسط یادگیری عمیق} 
		
		%% TODO: Adjust the size of the figure by using [scale=x] or [widht=.x\linewidth] (x is a fraction) to fit within the frame. Then rename to your picture file name and add the caption
	\begin{frame}{Deepfake}
		\begin{columns}
			\column{0.55\textwidth}
				\begin{itemize}
				\item  رسانه‌ها و شبکه‌های‌ اجتماعی
				\item اخبار غلط
				\item یادگیری عمیق
				\item \textbf{deepfake} ترکیب \ReverseWords{deep learning} و fake
				\item تشخیص دشوار
				\item ایجاد محدودیت‌های جدید
			\end{itemize}
			\hfill
			\column{0.5\textwidth}
				\begin{figure}
					\centering
					\includegraphics[width=0.7\linewidth,height=0.25\textheight]{images/screenshot001}
					\caption*{\scriptsize شکل ۲-۱: ویدئو deepfake از Barack Obama \\  \tiny \lr{https://www.youtube.com/watch?v=cQ54GDm1eL0}}
					\label{fig:screenshot001}
				\end{figure}
				\vspace{-0.7cm}
				\begin{figure}
					\centering
					\includegraphics[width=0.7\linewidth,height=0.25\textheight]{images/screenshot002}
					\caption*{\scriptsize شکل ۲-۲: ویدئو deepfake از Nicholas Cage \\ 
					\tiny \lr{https://www.youtube.com/watch?v=BU9YAHigNx8}}
					\label{fig:screenshot002}
				\end{figure}
				
		\end{columns}

	\end{frame}
		
	\begin{frame}{ویدئو‌ها از چهره افراد}
\begin{columns}
	
\column{0.55\textwidth}
\vspace{-0.2cm}
\begin{block}{\normalsize جابه‌جایی چهره (Face-swapping)}
	\begin{itemize}
		\item 	جابه‌جا کردن چهره فرد داخل ویدئو
		\item انتقال حالات صورت از فرد اصلی 
		\item  جا‌به‌جایی چهره با یک فرد مشهور
		\item \texttt{Faceswap-GAN}
	\end{itemize}
\end{block}
\vspace{-0.2cm}
\begin{block}{\normalsize بازسازی چهره (face-reenactment)}
	\begin{itemize}
		\item ثابت ماندن هویت فرد اصلی 
		\item انتقال حالات چهره به فرد اصلی
		\item به حرکت در آوردن چهره در عکس 
		\item \texttt{Face2Face}
	\end{itemize}
\end{block}
\column{0.5\textwidth}
\begin{figure}
	\centering
	\includegraphics[width=0.9\linewidth]{images/sensors-22-04697-g001-550}
	\caption*{\scriptsize شکل ۲-۳: مقایسه جابه‌جایی چهره و بازسازی چهره \tiny \lr{https://www.mdpi.com/sensors/sensors-22-04697/article-deploy/html/images/sensors-22-04697-g001-550.jpg}}
	\label{fig:sensors-22-04697-g001-550}
\end{figure}

\end{columns}
	\end{frame}


	
	
	
	\begin{frame}{مجموعه داده‌ها (Datasets)}
	\begin{columns}
		\column{0.5\textwidth}
		\begin{itemize}
			\item FaceForensics++
			\item ۸۰٪ برای یادگیری
			\item ۲۰٪ برای تست 
		\end{itemize}
		\column{0.5\textwidth}
		\begin{figure}
			\centering
			\includegraphics[width=0.9\linewidth]{images/screenshot003}
			\caption*{\scriptsize شکل ۲-۴: مشخصات مجموعه داده‌ها \cite{9302547}}
			\label{fig:screenshot003}
		\end{figure}
	\end{columns}
	
\end{frame}




\begin{frame}{پیش پردازش داده ها}
	\begin{columns}
		\column{0.5\textwidth}
		\begin{itemize}
			\item[\faExclamationCircle] عکس به عنوان ورودی مدل
			\begin{itemize}
				\item تبدیل ویدئو به فریم ها
				\item یک فریم از هر چهار فریم
			\end{itemize}
			\item[\faExclamationCircle] تشخیص بر اساس چهره
			\begin{itemize}
				\item  جداسازی چهره ها 
				\item \ReverseWords{cascade classifier}
			\end{itemize}
		\end{itemize}
		\column{0.5\textwidth}
		\begin{figure}
			\centering
			\includegraphics[width=0.85\linewidth]{images/screenshot004}
			\caption*{\scriptsize شکل ۲-۵: جدا سازی فریم و پردازش آنها \cite{9302547}}
			\label{fig:screenshot004}
		\end{figure}
		
	\end{columns}
	

\end{frame}	
\begin{frame}{مدل های یادگیری عمیق}
\begin{enumerate}
	\item \textbf{Xception}
	\begin{itemize}
		\item CNN
		\item معرفی توسط گوگل در سال ۲۰۱۷
		\item ۳۶ لایه کانولوشن در ۱۴ قسمت
		\item \ReverseWords{logistic regression} در لایه آخر  
	\end{itemize}
	\item \textbf{MobileNets}
	\begin{itemize}
		\item CNN
		\item سبک و بهینه
		\item ۲۸ لایه 
		\item پیاده سازی با TensorFlow	
	\end{itemize}
\end{enumerate}
\end{frame}	

\begin{frame}{نتایج}
	
	\begin{columns}
		\column{0.4\textwidth}
		\small
		\begin{itemize}
			\item ۸ مدل
			\item  عملکرد بهتر Xception
			\item دشواری تشخیص در NeuralTexture 
			\item تشخیص ویدئو مربوط به Obama
		\end{itemize}
		\column{0.8\textwidth}
		\begin{figure}
			\centering
			\includegraphics[width=1\linewidth]{images/screenshot005}
			\caption*{\scriptsize شکل ۲-۶: مقایسه نتایج مدل‌ها\cite{9302547}}
			\label{fig:screenshot005}
		\end{figure}
		
	\end{columns}
\end{frame}
\begin{frame}{‌چالش‌ها}
\begin{columns}
	\column{0.5\textwidth}
		\begin{itemize}
		\item[\faQuestionCircle] چرا مدل‌های جداگانه؟
		\item[\faExclamationCircle] ویدئو‌های مربوط به مدل‌های دیگر 
		\item[\faClipboardCheck] مکانیزم رائ‌گیری
	\end{itemize}
	\column{0.65\textwidth}
	\begin{figure}
		\centering
		\includegraphics[width=0.95\linewidth]{images/screenshot006}
		\caption*{\scriptsize شکل ۲-۷: بررسی مجموعه داده‌های دیگر با Deepfakes \cite{9302547}}
		\label{fig:screenshot006}
	\end{figure}
	
\end{columns}	

\end{frame}

\begin{frame}{قدمی های بعدی}
	\begin{itemize}
		\item[\faClipboardList] تغییر تابع هزینه و Optimizer
		\item[\faClipboardList] بررسی جداگانه اجزای صورت 
		\item[\faClipboardList] تشخیص بر اساس ویدئو
	\end{itemize}
\end{frame}
		%% TODO: Uncomment the next four lines to enable this frame
		% \begin{frame}{Methodology}
			% 	%% TODO: You can add the note here
			% 	\note{}
			% \end{frame}
		

		

\section{تشخیص فیشینگ توسط CNN} 
		\begin{frame}{ انواع مدل‌های تشخیص فیشینگ}
			\begin{figure}
				\centering
				\includegraphics[width=0.87\linewidth,height=0.5\linewidth]{images/fishing2}
				\caption*{\scriptsize شکل ۴-۱: نگاهی به روش های تشخیص فیشینگ \cite{electronics9091514}}
				\label{fig:fishing2}
			\end{figure}
		\end{frame}
		
		
		\begin{frame}{چرا تشخیص فیشینگ چالش برانگیز است؟}
			\begin{itemize}
				\item شباهت ظاهری سایت های جعلی
				\item ناشناخته بودن حملات روز صفر
				\item لیست های سیاه قابل دور زدن 
				\item نیاز به بررسی سریع و مستقل
			\end{itemize}
			\note{}
		\end{frame}
		\begin{frame}{راهکار شبکه های عصبی: مدل CNN بر پایه URL}
			\begin{itemize}
				\item[\faExternalLink*] ورودی فقط آدرس URL 
				\item[\faUserAltSlash] عدم نیاز به بررسی محتوای سایت
				\item[\faShield*] مقاوم در برابر حملات روز صفر
				\item[\faStopwatch] زمان تشخیص: 47.0 میلی ثانیه!
				\item[\faUnlink] مدل کاملا مستقل

			\end{itemize}
		\end{frame}
		
		
		\begin{frame}{ساختار مدل}
			\begin{itemize}
				\item بردار one-hot برای هر کاراکتر
				\item حداکثر ۲۰۰ کاراکتر
				\item ۷ لایه کانولوشن متوالی
				\item ۳ لایه \ReverseWords{fully connected} نهایی
				\item خروجی: فیشینگ/عادی
				
			\end{itemize}
		\end{frame}
		
		\begin{frame}{ساختار مدل (ادامه)}
			\begin{figure}
				\centering
				\includegraphics[width=0.87\linewidth,height=0.5\linewidth]{images/fishing1}
				\caption*{\scriptsize شکل ۴-۲: مروری بر مدل پیشنهادی \cite{electronics9091514}}
				\label{fig:fishing1}
			\end{figure}
			
		\end{frame}
		
		
		\begin{frame}{دقت مدل}
			\begin{itemize}
				\item دقت روی دیتاست مقاله: ۹۵٪
				\item دقت روی دیتاست خارجی: ۵۸٪.۹۸
				\item  بهتر از \ReverseWords{Random Forest} با دقت ۶۲٪.۹۳
				\item[\faClipboardCheck] سرعت و دقت در سطح بالا
				
			\end{itemize}
		\end{frame}
		
		\begin{frame}{جمع بندی و کاربرد ها}
			\begin{itemize}
			\item تشخیص سریع توسط URL خام
			\item کاربرد در مرورگر ها و فایروال ها
			\item مناسب برای مقابله با حملات جدید
			\item بدون نیاز به تخصص امنیتی
			\item ترکیب امنیت، یادگیری و سادگی 
			\end{itemize}
		\end{frame}

	

\appendix
	%% Thank You frame
	\begin{frame}
		\centering
		\includegraphics[width=.7\linewidth]{thankyou}
		%% TODO: You can add the note here
		\note{}
	\end{frame}
	\begin{frame}[allowframebreaks, noframenumbering]{مراجع}
		\begin{latin}
			\nocite{10127341}
			\printbibliography[heading=none]
		\end{latin}
		
	\end{frame}

	
\end{document}