\addcontentsline{toc}{section}{چکیده}
\newgeometry{left=2.5cm,right=3cm,top=3cm,bottom=2.5cm,includehead=false,headsep=1cm,footnotesep=.5cm}
\setcounter{page}{1}
\pagenumbering{arabic}
\thispagestyle{empty}

~\vfill

\subsection*{چکیده}
\begin{small}
	\baselineskip=0.7cm
	
	یادگیری عمیق\footnote{Deep Learning} به‌عنوان یکی از شاخه‌های پرکاربرد هوش مصنوعی، توانسته است در حوزه‌هایی مانند پزشکی، امنیت سایبری\footnote{Cybersecurity} و تحلیل محتوای دیجیتال، نتایج قابل توجهی به دست آورد. در یک پژوهش، یک شبکه عصبی کانولوشنی\footnote{Convolutional Neural Network - CNN} ۹ لایه برای تشخیص خودکار انواع ضربان‌های قلبی از روی سیگنال الکتروکاردیوگرام\footnote{Electrocardiogram - ECG} طراحی شد. این مدل توانست پنج نوع ضربان مختلف (از جمله ضربان‌های نارسایی بطنی و فوق‌بطنی) را حتی در شرایط نویزی با دقتی در حدود ٪۹۴ شناسایی کند. این مدل با استفاده از داده‌های تقویت‌شده و پاک‌سازی‌شده آموزش دید و توانست ابزار مناسبی برای غربالگری سریع آریتمی‌ها باشد.
	
	در حوزه رسانه‌های دیجیتال نیز، مطالعه‌ای به بررسی روش‌های تشخیص ویدیوهای جعلی (جعل عمیق)\footnote{Deepfake} با استفاده از دو معماری معروف Xception و MobileNet پرداخت. این مدل‌ها بر اساس داده‌های مجموعه‌ی FaceForensics++ آموزش دیدند که شامل ویدیوهایی با چهار روش مختلف ساخت جعل عمیق است. نتایج نشان داد که مدل‌ها دقتی بین ٪۹۱ تا ٪۹۸ دارند و استفاده از یک مکانیزم رأی‌گیری میان آن‌ها باعث افزایش دقت نهایی سیستم در تشخیص ویدیوهای جعلی شد.
	
	در حوزه امنیت سایبری نیز، یک مدل سریع و سبک برای تشخیص نشانی‌های فیشینگ\footnote{Phishing URLs} معرفی شد که بر پایه‌ی شبکه عصبی کانولوشنی در سطح کاراکتر عمل می‌کند. این روش برخلاف روش‌های سنتی نیازی به استخراج محتوای صفحات وب یا استفاده از سرویس‌های شخص ثالث ندارد. مدل با یادگیری الگوهای متنی موجود در نشانی وب، توانست بدون نیاز به ویژگی‌های دستی، نشانی‌های مخرب را با دقتی بالاتر از ٪۹۵ شناسایی کند و در چند پایگاه داده‌ی معتبر نیز دقتی بالاتر از روش‌های موجود به ثبت رساند.
	
	در مجموع، این مطالعات نشان می‌دهند که یادگیری عمیق با بهره‌گیری از معماری‌های مناسب و داده‌های غنی، می‌تواند در حل مسائل پیچیده در حوزه‌های حیاتی و چالش‌برانگیز عملکرد مؤثری داشته باشد.
	
	\vspace*{0.5 cm}
	
	\noindent\textbf{واژه‌های کلیدی:}
	
	تشخیص فیشینگ، مهندسی ویژگی‌های نشانی، نهفته‌سازی کاراکتری، یادگیری عمیق، تشخیص ضربان قلب، آریتمی، بیماری‌های قلبی، شبکه عصبی کانولوشنی، سیگنال الکتروکاردیوگرام
\end{small}
