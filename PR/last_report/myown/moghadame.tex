\chapter{مقدمه}

با گسترش سریع فناوری‌های مبتنی بر یادگیری عمیق\footnote{Deep Learning}، کاربردهای متنوعی در حوزه‌های مختلف از جمله امنیت سایبری\footnote{Cybersecurity}، تحلیل داده‌های زیستی، و پردازش سیگنال‌های پزشکی ایجاد شده است. در این میان، سه چالش مهم و نوظهور شامل تشخیص محتوای جعلی (جعل عمیق)\footnote{Deepfake}، شناسایی وب‌سایت‌های فیشینگ\footnote{Phishing Websites}، و تحلیل خودکار نوار قلب (الکتروکاردیوگرام)\footnote{Electrocardiogram - ECG} به‌عنوان موضوعات کلیدی پژوهش‌های اخیر مطرح شده‌اند.

جعل‌های عمیق، که با بهره‌گیری از شبکه‌های مولد تخاصمی\footnote{Generative Adversarial Networks - GANs} تولید می‌شوند، قابلیت شبیه‌سازی چهره و صدای افراد با دقت بالا را دارند و تهدیدی جدی برای اعتماد عمومی، انتخابات، و امنیت اطلاعات محسوب می‌شوند. در این زمینه، مدل‌های سبک و قدرتمند مانند Xception و MobileNet با استفاده از مجموعه‌داده‌ی FaceForensics++ جهت تشخیص ویدیوهای جعلی آموزش دیده‌اند و با ترکیب خروجی آن‌ها، مکانیزم رأی‌گیری برای افزایش دقت سیستم پیشنهاد شده است.

در حوزه امنیت سایبری، حملات فیشینگ همچنان از مهم‌ترین روش‌های سوءاستفاده مهاجمان برای سرقت اطلاعات شخصی کاربران محسوب می‌شود. بسیاری از روش‌های سنتی مانند لیست‌های سیاه یا تحلیل مبتنی بر خدمات شخص ثالث، در مقابله با حملات روز صفر\footnote{Zero-day Attacks} کارایی کافی ندارند. به همین دلیل، در این پژوهش از شبکه عصبی کانولوشنی\footnote{Convolutional Neural Network - CNN} در سطح کاراکتر برای تحلیل مستقیم رشته‌ی نشانی\footnote{َURL} استفاده شده است؛ روشی که مستقل از زبان، سریع، و بدون نیاز به مهندسی ویژگی دستی عمل می‌کند.

در حوزه پزشکی نیز بیماری‌های قلبی-عروقی، به‌ویژه آریتمی‌ها که ناشی از اختلال در سیستم الکتریکی قلب هستند، همچنان عامل اصلی مرگ‌ومیر در جهان‌اند. تشخیص دقیق این اختلالات از روی سیگنال‌های الکتروکاردیوگرام نیازمند تحلیل دقیق فرم موج و تفکیک انواع ضربان‌ها است. در این راستا، مدل پیشنهادی ما از یک شبکه عصبی پیچشی عمیق برای طبقه‌بندی خودکار ۵ نوع ضربان غیرعادی بهره می‌برد و توانسته است دقت بالایی را در داده‌های دارای نویز و بدون نویز ثبت کند.

ترکیب این سه مطالعه نشان می‌دهد که یادگیری عمیق با بهره‌گیری از معماری‌های بهینه‌ی کانولوشنی می‌تواند راهکارهای دقیق، سریع و مقیاس‌پذیر برای حل مسائل پیچیده در حوزه‌های پزشکی، امنیت دیجیتال و رسانه فراهم آورد.
