\chapter{نتیجه‌گیری}

در این پژوهش، کاربرد یادگیری عمیق در سه حوزه حیاتی شامل تشخیص محتوای جعلی (جعل عمیق)، شناسایی حملات فیشینگ، و تحلیل سیگنال‌های قلبی مورد بررسی و پیاده‌سازی قرار گرفت. برای هر مسئله، معماری‌هایی متناسب انتخاب شد که نشان‌دهنده توانایی بالای شبکه‌های عصبی در استخراج ویژگی‌های مؤثر از داده‌های خام و افزایش دقت تشخیص هستند.

در حوزه تشخیص جعل عمیق، مدل‌های \lr{Xception} و \lr{MobileNet} به‌طور مستقل و ترکیبی عملکرد مناسبی در شناسایی ویدیوهای جعلی حاصل از چهار تکنیک رایج نشان دادند. ترکیب این مدل‌ها با سازوکار رأی‌گیری منجر به افزایش دقت کلی و کاهش خطاهای تک‌مدل‌ها شد. آینده این حوزه می‌تواند شامل آموزش مدل‌های جدید برای روش‌های نوین تولید جعل عمیق و تحلیل فریم‌های زمانی به‌جای تصاویر ایستا باشد.

در زمینه مقابله با فیشینگ\footnote{Phishing}، استفاده از شبکه عصبی کانولوشنی\footnote{Convolutional Neural Network} در سطح کاراکتر توانست بدون تکیه بر خدمات شخص ثالث و مهندسی ویژگی دستی، نشانی‌های\footnote{URL} جعلی را با دقت مناسب تشخیص دهد. این روش، زمان پاسخ بسیار کمی دارد و برای پیاده‌سازی سمت کلاینت مناسب است. به‌منظور بهبود عملکرد، پیشنهاد می‌شود از محتوای HTML صفحات و کدهای سمت کاربر نیز در طراحی مدل‌های آینده بهره گرفته شود.

در حوزه پزشکی، مدل کانولوشنی توسعه‌یافته توانست ۵ نوع ضربان غیرعادی قلب را با دقت بالا طبقه‌بندی کند و به‌عنوان ابزار پشتیبان تصمیم در سیستم‌های تشخیص یاری‌رسان پزشکی (\lr{CAD}\footnote{Computer-Aided Diagnosis}) مطرح گردد. این سیستم نه‌تنها موجب کاهش خطاهای انسانی می‌شود، بلکه توانایی استفاده در محیط‌های بالینی برای غربالگری سریع سیگنال‌های ECG را نیز دارد. گسترش آینده شامل تحلیل دنباله‌های زمانی ضربان، بررسی عملکرد مدل در داده‌های متعادل‌شده و نویزی و همچنین تفکیک بیماران در سه سطح خطر خواهد بود.

در مجموع، نتایج حاصل از این مطالعه چندجانبه نشان می‌دهد که یادگیری عمیق، در کنار معماری‌های مناسب و انتخاب ویژگی‌های درست، می‌تواند راه‌حلی اثربخش و کاربردی برای حل مسائل پیچیده در حوزه‌های امنیتی، زیستی و اجتماعی باشد.
